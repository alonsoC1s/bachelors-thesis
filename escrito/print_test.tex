\documentclass[colorful]{sty/itam-thesis}

\usepackage{lipsum}

\usepackage[
		showframe,
		paperwidth=17cm,
		paperheight=22.5cm,
		nofoot=true,
		bindingoffset=1.1cm,
		inner=1.6cm,
		outer=1.8cm,
		top=2.5cm,
		bottom=1.5cm
	]{geometry}

\usepackage{tikz}
\usepackage{tikzscale}
\usetikzlibrary{backgrounds, intersections, fillbetween}
% \usetikzlibrary{external}
% \tikzexternalize[prefix=tikz/]
\usepackage{import}

% Bibliografía
\usepackage{csquotes}
\usepackage[
	backend=biber,
    citestyle=alphabetic,
    style=alphabetic,
    maxcitenames=2,
    ]{biblatex}
\addbibresource{BachelorsThesis.bib}

% Paquete custom
\usepackage{sty/thesis-package}

\author{Alonso Martinez Cisneros}
\title{%
	Soluciones aproximadas al problema de Aprendizaje por Refuerzo mediante
	Programación lineal; con aplicaciones al Aprendizaje de Máquina Supervisado%
}
\subtitle{%
	Approximate solutions to the Reinforcement Learning Problem via Linear
	Programming; with applications to Supervised Machine Learning.%
}
\date{2022}

\begin{document}

\frontmatter
\pagenumbering{roman}
\maketitle
\makefrontmatter



\tableofcontents

\mainmatter

%====================================================%
% Start of the content proper ---------------------- %
%====================================================%
\pagestyle{plain}
\pagenumbering{arabic}

\part{Primera parte}
\chapter{Primer capítulo}

\section{Sección Primera}
\lipsum[1-2]

\subsection{Subsección}
Un poco de matemáticas:

A \emph{policy} is a mapping from states to actions $\pi: S \to A$.

\begin{dfn}{Value function}{}
	A \emph{value function} from a policy, written $V^{\pi}: S \to \R$ gives the
	expected sum of discounted rewards when acting under that policy
	\begin{equation}
		V^{\pi} (s) = \E \left[ R_{t+k+1} \vertsep A \right].
	\end{equation}
\end{dfn}

Ahora referenciamos a \cite{farias2003LP2ADP} para probar el hightlighting de cita. Ahora pruebo 
los theorem-like environments.

\begin{thrm}{Borel--Carathéodory}{borel-cara}
Let a function $f$ be analytic on a \emph{closed disc} of radius $R$ center origin.
Suppose that $r < R$. Then, we have the following inequality:
\begin{equation}
\|f\|_r \le \frac{2r}{R-r} \sup_{|z| \le R} \operatorname{Re} f(z) + \frac{R+r}{R-r} 
|f(0)|.
\end{equation}
\end{thrm}

\lipsum[2][2]

\begin{thrm}{Liouville}{liouville}
	Every holomorphic function $f$ for which there exists a positive 
	number $M$ such that $|f(z)| \leq M$ for all $z$ in $\mathbb{C}$
	is constant.
\end{thrm}


\begin{coro}{}{}
	Non-constant holomorphic functions on $\mathbb{C}$  have unbounded 
	images.
\end{coro}

\begin{lemma}{}{}
	Every non-constant single-variable polynomial with complex 
	coefficients has at least one complex root.
\end{lemma}

\subsubsection{Prueba de código}
Finalmente, un poco de código.

\begin{lstlisting}[language=julia, caption=Aplicando algoritmo de cifrado]
# Definiendo sistema de Lorenz
function lorenz!(du, u, p, t, σ=10, β=8//3, ρ=28)
    du[1] = σ * (u[2] - u[1])
    du[2] = u[1] * (ρ - u[3]) - u[2]
    du[3] = u[1] * u[2] - β * u[3]
end
\end{lstlisting}

\lipsum[3][1-3]

\begin{lemma}{}{}
	Let $\displaystyle \mathbf{A} \in \mathbb{R}^{m\times n}$ and 
	$\displaystyle \mathbf{b} \in \mathbb {R}^{m}$. Then exactly one 
	of the following two assertions is true:
	\begin{enumerate}
		\item There exists an $\displaystyle \mathbf{x} \in \mathbb 
			{R} ^{n}$ such that $\mathbf{Ax} =\mathbf{b}$ and 
			$\displaystyle \mathbf{x} \geq 0$.
		\item There exists a $\displaystyle \mathbf{y} \in \mathbb
			{R}^{m}$ such that $\displaystyle 
			\mathbf{A}^{\top}\mathbf{y} \geq 0$ and $\displaystyle 
			\mathbf{b}^{\top}\mathbf{y} <0$.
	\end{enumerate}
\end{lemma}

\begin{algorithm}
    \SetKwFunction{Predict}{predict}
    \KwIn{The tree $f_\L$}
    \KwIn{The feature vector $\vec{x}$}
    \KwOut{The predicted class $\widehat{y}$ for $\vec{x}$}
    \Function{\Predict{$f_\L, \vec{x}$}}{
        $t \gets t_0$ \;
        \While{$t$ is not a terminal node}{
            $t \gets$ the child node $t'$ of $t$ such that $\vec{x} \in \mathcal{X}_t$ \;
        }
        \Return{$\widehat{y}_t$} \;
    }
    \caption[Tree prediction algorithm]{Predict output value $\widehat{y}$ with
        tree $f_\L$ \cite[Ch.~3.2]{louppe2014}.}
    \label{alg:tree-predict}
\end{algorithm}

\begin{figure}
\centering
\includegraphics[width=\linewidth]{img/log.tikz}
\caption{fig:pruebas}
\end{figure}

\printbibliography

\end{document}