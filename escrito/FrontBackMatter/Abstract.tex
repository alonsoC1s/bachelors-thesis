\addcontentsline{toc}{chapter}{Abstract}

\chapter*{Resumen}
En las últimas décadas avances en el campo conocido como Inteligencia Artificial
han cambiado el panorama de que problemas tienen y no tienen solución. Problemas
como procesamiento de lenguaje natural, reconocimiento de imágenes y otros,
ahora es posible resolver y son utilizados en nuestro día a día. Estos avances
se han dado gracias a la sinergia entre las Matemáticas y las Ciencias de la
Computación.

Estas herramientas pasaron de ser problemas abiertos en a investigación, a ser
problemas de ingeniería que se resuelven a escala día tras día. Una gran parte
de las personas involucradas en la implementación de estas soluciones tienen
conocimiento limitado sobre los conceptos matemáticos que subyacen la práctica
que llevan a cabo, lo cual es natural dado que no es conocimiento esencial para
sus tareas cotidianas. Esta tésis es un esfuerzo por explorar la teoría y las
intuiciones detrás de una de las técnicas más novedosas de la Inteligencia
Artificial: Aprendizaje por Refuerzo (\textsc{rl} por sus siglas en inglés).
Esta tesis busca sentar las bases en Procesos Estocásticos, Probabilidad, y
Optimización que dan el trasfondo para entender el aprendizaje por refuerzo. Se
busca presentar el contenido de manera intuitiva sin dejar de lado el rigor
matemático. Como matemático siento una obligación por presentar el campo de
estudio con la belleza que percibimos quienes lo estudiamos.

Esta tesis está compuesta por capítulos divididos en tres partes. La primera
parte presenta las ideas fundamentales de los diferentes campos de las
matemáticas que son esenciales para el aprendizaje por refuerzo, y una corta
introducción al campo del aprendizaje de máquina supervisado. La segunda parte
presenta en concreto lo que llamaremos el \emph{problema de aprendizaje por
refuerzo} en términos de un problema matemático, y desarrollamos la teoría que
permite dar soluciones aproximadas a este problema explicando porqué son de
interés. La tercera y última parte establece la conexión entre el aprendizaje de
máquina supervisado con el aprendizaje por refuerzo, y propone un algoritmo
basado en las soluciones aproximadas que son el foco principal de esta tésis al
problema de ajuste de árboles de decisión.

\chapter*{Abstract}
In the last 20 or so years, several advancements and novel techniques have
transformed the landscape of the discipline we currently call Artificial
Intelligence. These new approaches have made possible tasks deemed intractable
decades prior, such as natural language processing, image recognition, and
human-level competence at certain games, to name a few. These advancements have
come from a fruitful synergy between several fields of study: Mathematics and
Computer Science, to be precise.

As these techniques have moved from being the \textit{state of the art} to
mainly becoming a problem of engineering, most the people currently implementing
solutions based on Artificial Intelligence today have limited knowledge of the
mathematical underpinnings that enable such powerful methods, for they do not 
need it to do their job. This thesis represents an effort to explore the theory
and intuitions behind one of the most innovative techniques in Artificial
Intelligence: \acf{rl}. This work aims to explore the key ideas in
the areas of mathematics that provide the foundations for Reinforcement
Learning: Stochastic Processes, Probability Theory, and a particular emphasis on
Mathematical Optimization, so the field and problems can be presented in an
engaging fashion while sacrificing as little clarity as possible.

This thesis consists of mainly three parts, and several chapters that make up
said parts. The first part presents the main ideas from the different fields of
mathematics that will be needed to motivate and justify the theory behind
\ac{rl}. The second part explores the \ac{rl}
problems as a matter of Mathematics and how these problems are solved through
different optimization techniques. The third and final part deals with
applications of \ac{rl}.
