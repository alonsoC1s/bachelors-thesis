\documentclass[11pt, oneside]{book}
\usepackage[
		paperwidth=17cm,
		paperheight=22.5cm,
		bottom=2.5cm,
		right=2.5cm
	]{geometry}

% El borde inferior puede parecerles muy amplio a la vista. Les recomiendo hacer una prueba de impresión antes para ajustarlo

\usepackage{
		amssymb,
		amsmath,
		amsthm,
		mathtools,
		nccmath
	}

% \usepackage[spanish]{babel}
% \usepackage[utf8]{inputenc} % Acentos y otros símbolos 
\usepackage{enumerate}
\usepackage{hyperref} % Hipervínculos en el índice
\usepackage{graphicx}
%\usepackage{subfig} % Subfiguras
\graphicspath{{img/}} % En qué carpeta están las imágenes

\DeclarePairedDelimiter{\nint}\lfloor\rceil %nearest integer
\DeclarePairedDelimiter{\ceil}\lceil\rceil 
\DeclarePairedDelimiter{\floor}\lfloor\rfloor 
\DeclarePairedDelimiter{\abs}\lvert\rvert % absolute value

% Para eliminar guiones y justificar texto
% \tolerance=1
% \emergencystretch=\maxdimen
% \hyphenpenalty=10000
% \hbadness=10000

% \linespread{1.25} % Asemeja el interlineado 1.5 de Word

\let\oldfootnote\footnote % Deja espacio entre el número del pie de página y el inicio del texto
\renewcommand\footnote[1]{%
\oldfootnote{\hspace{0.05mm}#1}}

\renewcommand{\thefootnote} {\textcolor{Black}{\arabic{footnote}}} % Súperindice a color negro

\setlength{\footnotesep}{0.75\baselineskip} % Espaciado entre notas al pie

% \usepackage{fnpos} % Footnotes al final de pág.

% \usepackage[justification=centering, font=bf, labelsep=period, skip=5pt]{caption} % Centrar captions de tablas y ponerlas en negritas

% \newcommand{\imagesource}[1]{{\footnotesize Fuente: #1}}

\usepackage{tabularx} % Big tables
\usepackage{graphicx}
\usepackage{adjustbox}
\usepackage{longtable}

\usepackage{float} % Float tables

\usepackage[usenames,dvipsnames]{xcolor} % Color

\usepackage{pgfplots} % Gráficas
\pgfplotsset{compat=newest}
\pgfplotsset{width=7.5cm}
\pgfkeys{/pgf/number format/1000 sep={}}


\DeclareMathOperator*{\argmax}{arg\,max}
\DeclareMathOperator*{\argmin}{arg\,min}

% Cosas para algoritmos
\usepackage[ruled, vlined, linesnumbered]{algorithm2e}

\theoremstyle{definition}
\newtheorem{definition}{Definición}[section]

\begin{document}

%----------------------------------------------------------------------------------------
%	PORTADA
%----------------------------------------------------------------------------------------

% Estableciendo variables que luego se usan en title page
\title{Markov Decision Processes and Applications to Reinforced Learning}
\author{Alonso Martinez Cisneros}
\date{2022}

\makeatletter
\begin{titlepage}
\begin{center}

\textsc{\Large Instituto Tecnológico Autónomo de México}\\[2em]

% Logo ITAM
\begin{figure}[h]
\begin{center}
\includegraphics[scale=0.50]{logo-ITAM.pdf}
\end{center}
\end{figure}

\textbf{\LARGE \@title }\\[2em]

\textsc{\large Tesis}\\[1em]

\textsc{\large que para obtener el título de}\\[1em]

\textsc{\LARGE Licenciado en Matemáticas Aplicadas}\\[1em]

\textsc{\large Presenta}\\[1em]

\textsc{\LARGE \@author}\\[1em]

\textsc{\large Asesor}\\[1em]

\textsc{\LARGE Dr. Andreas Wachtel}\\[2em]

\end{center}

\vspace*{\fill}
\textsc{Ciudad de México \hspace*{\fill} \@date}

\end{titlepage}
\makeatother

%----------------------------------------------------------------------------------------
%	DECLARACIÓN
%----------------------------------------------------------------------------------------

\thispagestyle{empty}

\vspace*{\fill}
\begingroup

\noindent
\makeatletter
«Con fundamento en los artículos 21 y 27 de la Ley Federal del Derecho de Autor y como 
titular de los derechos moral y patrimonial de la obra titulada ``\textbf{\@title}'', 
otorgo de manera gratuita y permanente al Instituto Tecnológico Autónomo de México y a la 
Biblioteca Raúl Bailléres Jr., la autorización para que fijen la obra en cualquier medio, 
incluido el electrónico, y la divulguen entre sus usuarios, profesores, estudiantes o 
terceras personas, sin que pueda percibir por tal divulgación una contraprestación.»

% Asegúrense de cambiar el título de su tesis en el párrafo anterior

\centering 

\vspace{5em}

\rule[1em]{20em}{0.5pt} % Línea para la fecha

\textsc{Fecha}
 
\vspace{8em}

\rule[1em]{20em}{0.5pt} % Línea para la firma

\textsc{\@author}

\endgroup
\vspace*{\fill}
\makeatother

%----------------------------------------------------------------------------------------
%	DEDICATORIA
%----------------------------------------------------------------------------------------

\pagestyle{plain}
\frontmatter

\chapter*{}
\begin{flushright}
\textit{A mis padres,\\ por su incansable esfuerzo.}
\end{flushright}

%----------------------------------------------------------------------------------------
%	AGRADECIMIENTOS
%----------------------------------------------------------------------------------------

\chapter*{Agradecimientos}

\noindent Lorem ipsum dolor sit amet, consectetur adipiscing elit.

% Esta sección es lo único que la gente lee. True story :)

%----------------------------------------------------------------------------------------
%	TABLA DE CONTENIDOS
%---------------------------------------------------------------------------------------

% \renewcommand{\contentsname}{Tabla de contenido}
\tableofcontents

%----------------------------------------------------------------------------------------
%	TESIS
%----------------------------------------------------------------------------------------

\mainmatter % Empieza la numeración de las páginas

\pagestyle{plain}

%----------------------------------------------------------------------------------------
%	BIBLIOGRAFÍA
%----------------------------------------------------------------------------------------



\end{document}
