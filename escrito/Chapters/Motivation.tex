\section{A ``hand-wavy'' first approach}

What we refer to as Artificial Intelligence is a broad collection of 
methods and techniques used to solve a wide array of problems that we 
collectively associate with ``human intelligence'', such as 
identifying and classifying images, processing natural language, and 
learning from data. Roughly speaking, the discipline can be divided in 
the following sub-fields of research:

\begin{itemize}
	\item Neural Networks.
	\item Vision.
	\item Natural Language Processing.
	\item Speech processing.
	\item Machine Learning.
\end{itemize}

This thesis is concerned specifically with Machine Learning.  
Particularly with a subset of Machine Learning known as unsupervised 
learning.  Some authors \cite{SuttonBarto} insist Reinforcement 
Learning is itself different from unsupervised learning, but for the 
sake of simplicity we address the distinctions later on.

In contrast to supervised learning where a learning agent is given 
data labeled by a knowledgeable source and must ``learn'' to classify 
based on those initial labels, in unsupervised learning there is no 
``train data'', the agent must act and optimize it's strategy based 
only upon the reward or penalty resulting from making a certain 
decision. The key words here being \textit{optimize}, 
\textit{strategy} and \textit{decision}.

For a concrete example, picture a robot vacuum cleaner tasked with 
moving from an initial point to a target point.  Only that this robot 
has no sense of direction, it can only move up, down, left or right 
with respect to itself. The robot cannot ``tell'' which way to go, it 
only senses if got closer or farther away. Is there a way the robot 
can reach it's goal?

If you think about it, the robot analogy is not so different from the 
standard mental model we have of say an infant learning to crawl. With 
poor vision and only it's caretaker's voice as guide, it must learn to 
find its way to safety by trial and error. It cannot be given millions 
of examples of ``valid paths'' for extrapolation, as is the case with 
supervised learning. This agent learns by interacting with the 
environment itself.

In a certain sense, Reinforcement Learning leverages our intuitions 
about the nature of learning. All the main elements are there: cause 
and effect, goals, and consequences to decisions made, but as we will 
see, it also encodes more subtle concepts. For instance, delayed 
gratification and planning ahead. It also has the novelty of being 
goal-oriented rather than task-oriented as most Machine Learning 
techniques often are. For example, a self-driving car ``trained'' via 
supervised learning might train on millions of examples on what 
constitutes a valid steering wheel move, while one learning without 
supervision is learning how to drive as an activity consisting of 
hundreds of little tasks, all to be mastered.

\section{Formalizing ideas}
Now, mathematically speaking, what does it \textit{mean} for a machine 
to ``learn''?

We might not get as far as what learning as a whole means for humans 
or machines, but we can certainly discuss what mechanisms allow for 
things such as self-driving cars and computers beating world champions 
of Chess and Go. For that, we need to identify the key concepts in the 
picture I painted and express them in the language of mathematics.

For now, we can gloss over the details of how might a machine make 
decisions, perceive goals, take actions or perceive rewards. Let's 
focus on one key component: the ``learning''. When we talk about a 
machine learning something we are thinking about some agent that is 
able to keep track of the decisions it made before and whether or not 
they resulted in positive results so it can later on apply that 
knowledge to become an increasingly better problem solver. If this 
agent is scored each time it carries on some task, we would like for 
it's score to increase each successive time it tries to complete the 
task. The idea of a continually improving score is the foundation of 
mathematical optimization.

\subsection{Optimization Theory}
As the name suggests, mathematical optimization is concerned with 
finding the ``optimal'' solution to problems where an unambiguous 
score might be given to different solutions. Whatever ``optimal'' 
means. Even if there is no best solution to a given problem, the 
techniques used in mathematical optimization often allow for a 
continual improvement through iteration.

Going back to our example of a robot learning how to drive, each time 
this robot runs a particular course it receives a certain grade. For 
instance, crashing or going off the circuit result in a penalty 
(negative grade) and respecting traffic signals results in a reward.  
If we want the robot to become an increasingly better driver we must 
frame the process of successive tries as an optimization problem. If 
we are able to solve this problem and achieve a continually improving 
grade, our robot will in a sense be learning to be a better driver.

We are now ready to peel away another layer of abstraction. We now 
have an intuition of what it means for a machine to learn, or in other 
words continually improve. But in the model we discussed earlier, this 
robot is making choices along the way. A predefined path is not 
programmed or even known. What does it mean for this robot to ``make 
choices''?

\subsection{Stochasticity}
Another key aspect we take for granted when we talk about machines 
learning is implicit in the word leaning. If there were a predefined 
path for the robot to take we would hardly call that learning, it's 
merely reproducing instructions. What this means for us, is that we 
must allow for a framework in which our robots actions are not 
completely determined beforehand. In fancier words, the succession of 
events that determine the robots actions and responses to stimuli are 
not \textit{deterministic}.

This idea is formalized through something called a \textit{stochastic 
process}. The word stochastic is just mathematician speak for ``not 
entirely determined beforehand'' or ``aptly described as a random 
phenomena''. In particular, we can think of the robot as being in some 
state among many many possible. The robot can change states as time 
goes by but the transitions are not always the same and they don't 
always carry the same reward or punishment.

For instance, on a certain circuit speeding up is bad, while on others 
its appropriate. You wouldn't drive at the same speed on a street as 
you do on a highway, even if they are both straight paths. If the 
robot were learning anything it would know this. In both cases its 
current state is ``driving down a straight path'', but transitions to 
a state of breaking or accelerating for both scenarios are not equally 
likely, nor they should be.

\section{Wrapping up}
So far we have developed a mental model of what to do should we wish 
to teach a robot how to drive. The rest of this thesis is dedicated to 
the careful development of the ideas here presented into the language 
of mathematics. But beyond mere description, this exercise has the 
potential to unlock \textit{insight}. As is often said by legendary 
math communicator Grant Sanderson\footnote{From the YouTube channel 
\href{https://www.youtube.com/channel/UCYO_jab_esuFRV4b17AJtAw}{3blue1brown}}(loose 
quote), the point in formulating things this way is to gain a deeper 
understanding of the phenomenon. So let's dive right in.
