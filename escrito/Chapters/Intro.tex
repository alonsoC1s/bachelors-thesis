\chapter*{Introduction}
\label{chapter:Intro}

In the last 20 or so years, several advancements and novel techniques have
transformed the landscape of the discipline we currently call Artificial
Intelligence. These new approaches have made possible tasks deemed intractable
decades prior, such as natural language processing, image recognition, and
human-level competence at certain games, to name a few. These advancements have
come from a fruitful synergy between several fields of study: Mathematics and
Computer Science, to be precise.

As these techniques have moved from being the \textit{state of the art} to
mainly becoming a problem of engineering, most the people currently implementing
solutions based on Artificial Intelligence today have limited knowledge of the
mathematical underpinnings that enable such powerful methods, for they do not 
need it to do their job. This thesis represents an effort to explore the theory
and intuitions behind one of the most innovative techniques in Artificial
Intelligence: \ac{rl}. This work aims to explore the key ideas in
the areas of mathematics that provide the foundations for Reinforcement
Learning: Stochastic Processes, Probability Theory, and a particular emphasis on
Mathematical Optimization, so the field and problems can be presented in an
engaging fashion while sacrificing as little clarity as possible. Then, some 
selected problems and applications will be presented to illustrate the power of
the techniques the reader has become acquainted to. 

As a mathematician, I feel a special obligation to share the beauty of the ideas
in our field with those outside of it. To that end, to make this work more
enjoyable to the people I dedicate it to, I try to emulate a more leisurely
style than the one found in research papers.  Some terseness will be sacrificed
in favor of clarity. Nevertheless, the mathematical minutia will not be ``swept
under the rug'' or separated from the main body of text. Instead, I will try to
give context to the development of mathematical ideas while not insisting on
subjecting the reader to every single technical detail necessary for the
formality of the arguments. The involved details will be available to those
curious in the appendixes.

This thesis consists of mainly three parts, and several chapters that make up
said parts. The first part presents the main ideas from the different fields of
mathematics that will be needed to motivate and justify the theory behind
\ac{rl}. The second part explores the \ac{rl}
problems as a matter of Mathematics and how these problems are solved through
different optimization techniques. The third and final part deals with
applications of \ac{rl}.

Interspersed among the purely mathematical theory are examples of how these
ideas are expressed in the \href{https://julialang.org/}{Julia Programming
Language}. These code snippets will be primarily self-contained, ready to run
and leverage the vast ecosystem that makes Julia ideal for Mathematics and
Scientific disciplines. Some code examples will play the role of pseudo-code as
Julia's simple syntax allows the mathematical ideas to shine through the
implementation details.