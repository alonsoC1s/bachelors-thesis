\chapter{Conclusions and Future Work}
\label{chapter:Conclusions}

The fields of Machine Learning, Reinforcement Learning, Data Science and others,
commonly grouped together under the term Artificial Intelligence; are changing
the landscape of a vast number of disciplines at an ever-increasing pace. In the
last decade or so, problems such as computer vision and self-driving have gone
from being open research problems to engineering problems. To sustain these
advances it is important to keep investing not only in reasearch from an
academic standpoint, but also to allow practicioners to access the tools to
better understand the underlying principles that underpin the techniques they
use on a daily basis to solve problems of the industry. This thesis makes an
effort to present the concepts needed to understand the mathematical framework
of Reinforcement Learning.

In addition to a detailed review of the theory, we explore some of the more
recent techniques in the field developed in the last decades, and propose how
Reinforcement Learning might be useful in other areas of Artificial
Intelligence. This thesis culminates by making the connection between these
seemingly disconnected areas an idea that is, if not obvious, organic. Sadly,
the implementation of \acf{alpbdt} and the proper comparison with \acf{rlbdt}
could not be included in the present work due to several constraints. Hopefully,
an implementation and more serious evaluation can be undertaken as a part of
future work.

\section{Future work}
As mentioned previously, the \ac{alpbdt} algorithm could not be fully
implemented, as the Approximate Linear Programming approach to solving a general
Reinforcement Learning problem requires problem-specific knowledge to
effectively model and leverage the strengths of the approach. There are many
ways to overcome those constraints, one of which is simply investing more time
and effort to gain that problem-specific knowledge. Sadly, this is outside of
scope for a bachelor's thesis, and would require a dedicated and in-depth study
of the particular problem. With this in mind, the completion of the
implementation is part of my future plans for future academic endeavors.

Beyond Reinforcement Learning, \ac{alpbdt} could be of significant importance
specifically in the field of supervised machine learning. As was proved by
\cite{caglar22}, any neural network with piece-wise linear activation functions
can be represented \emph{exactly} as a decision tree, no approximations
required. In the context of our work this hints that, not only \ac{rlbdt}'s
prediction performance \cite{xiong} can be matched, there is some hope to
achieve this parity at a much lower computational cost for the training
procedure. Looking further, this means that neural networks, the bleeding edge
technique in the field, can be understood and represented as the simple
structure reviewed in this thesis. Understanding decision trees is becoming
more relevant than ever, and finding better ways to leverage their power might
be worthwile path for future reaserchers to follow.