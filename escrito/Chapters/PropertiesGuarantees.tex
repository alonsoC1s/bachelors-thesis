\chapter{Properties and performance guarantees}
\label{chapter:PropertiesGuarantees}

As previously mentioned, the Approximate Linear Programming (ALP) approach to
solving the RL problem is based on work by \citeauthor{farias2003LP2ADP}. This
chapter is devoted to reproducing and discussing some of the main results
presented in \cite{farias2003LP2ADP}. The proofs are left out in favour of more
verbose explanations of what the results might entail. This chapter aims to
demistify some of the claims made about linear program \eqref{lp:approx-lp}
being ``easier'' to solve.

Recall that in the definition of \eqref{lp:approx-lp} we referred to the vector
$\vec{c}$ as the \emph{state-relevance wights}. The choice of state-relevance
weights does not influence the solution of \eqref{lp:exact-lp}, but it does
affect \eqref{lp:approx-lp}. The results discussed below demonstrate the impact
on the quality of the resulting approximation.

\section{Preliminaries}

\begin{dfn}{Vector-Weighted $\ell_1$ norm}{vectorw-l1-norm}
    The \emph{vector-weighted} 1-norm over the space $\ell_1$, denoted $\left\|
    \cdot \right\|_{1, \vec{c}}$, of a vector $\vec{x}$ is defined as
    \[
        \left\| \vec{x} \right\|_{1, \vec{c}} \coloneqq  \sum_i |x_i| c_i.
    \]
    Vector $\vec{c}$ must be non-negative in each entry, and the same size as
    $\vec{x}$.
\end{dfn}

We defined the weighted $\ell_1$ norm for vectors, but it will become useful to
define it for probability distributions. This is justified; since, as reviewed
in chapter \ref{chapter:ApproximateLinearP}, vectors $\vec{v}_*$ and
$\vec{v}_\pi$ are column vectors in which each entry is the state-value function
evaluated at a single state $s$. By extension, for a distribuiton $\sigma$, we
can imagine that the vector $\sigma$ is a column vector where each entry is the
probability of some $s \in \States$ happening. Formally, we define this
extension of the weighted norm as follows.

\begin{dfn}{Distribution-Weighted $\ell_1$ norm}{distributionw-l1-norm}
    Extending the vector-weighted $\ell_1$ norm we define the norm
    \[
        \left\| \vec{v} \right\|_{1, \sigma} \coloneqq \sum_{s \in \States} \sigma(s) \, |v(s)|,
    \]
    for some vector $\vec{v} \sim \sigma$ where $\sigma$ is a probability
    distribution.
\end{dfn}

To measure the quality of a specific policy $\pi$ we will consider the how the
value $v_\pi(s)$ compares to the optimal value $v_* (s)$ when the initial state
$s$ is a random variable with probability distribution $\sigma$. Intuitively,
how far are the expected total discounted rewards from the optimal when
following policy $\pi$.

\begin{dfn}{Expected increase in value following $\pi$}{expected-value-increase}
    The expected increase in value following a policy $\pi$ is defined as
    \begin{equation}
        \label{eq:expected-value-increase}
        \E_{s \sim \sigma} \left[ v_* (s) - v_\pi (s) \right] = \left\| \vec{v
        }_* - \vec{v}_\pi \right\|_{1, \sigma}.
    \end{equation}
    The notation $s \sim \sigma$ means that $s$ is a particular realization of the
    random variable $S$, which is distributed according to $\sigma$.
\end{dfn}

Next, we define a probability measure that captures the probability of the agent
being in some state given that it is following policy $\pi$ and starte on some
randomly distributed $s \sim \sigma$.

\begin{dfn}{The $\mu$ measure}{mu-measure}
    We define a measure represented by the vector $\mu$, where each entry
    $\mu(s)$ is the measure evaluated at that $s$, as
    \[
        \mu_{\pi, \, \sigma}^{\top} \coloneqq (1 - \gamma) \sigma^{\top} \sum_{t=0}^{\infty} \gamma^{t} \vec{P}_{\pi}^{t}.
    \]
    Since $\sum_{t=0}^{\infty} \gamma^{t} \vec{P}_{\pi}^{t} = (I - \gamma
    \vec{P}_\pi)^{-1}$, where $I$ is the identity matrix, we have
    \[
        \mu_{\pi, \, \sigma}^{\top} = (1 - \gamma) \sigma^{\top} (I - \gamma \vec{P}_{\pi})^{-1}.
    \]
    We say $\mu$ is a measure in the sense of measure theory. It can be shown
    \Cite[pg.~864]{farias2003LP2ADP} that $\mu$ is a probability distribution.
\end{dfn}

\section{Error bounds for the ALP}

We begin with a lemma that helps illustrate the role of state-relevance weights
for the approximation procedure.

\begin{lemma}{}{farias-vanroy-lem1}
    A vector $\vec{\beta}_0$ solves the following LP
    \[
    \begin{array}{rl@{}ll}
        \displaystyle \min_{\vec{\beta} \in \R^{K}} & \vec{c}^{\top} \Phi \vec{\beta} \\
        \text{S.t.} & \displaystyle \Phi \vec{\beta} (s) \geq r(s, a) + \gamma \sum_{s'} p(s' \mid s, a) \Phi \vec{\beta}(s') & \quad \forall a, s \in \Actions, \States . \\
    \end{array}
    \]
    if and only if it solves
    \[
    \begin{array}{rl@{}ll}
        \displaystyle \max_{\vec{\beta} \in \R^{K}} & \left\| \vec{v}_* - \Phi \vec{\beta} \right\|_{1, \, \vec{c}} \\
        \text{S.t.} & \displaystyle \Phi \vec{\beta} (s) \geq r(s, a) + \gamma \sum_{s'} p(s' \mid s, a) \Phi \vec{\beta}(s') & \quad \forall a, s \in \Actions, \States . \\
    \end{array}
    \]
\end{lemma}

Lemma \ref{lem:farias-vanroy-lem1} corresponds to Lemma 1 in
\Cite[pg.~853]{farias2003LP2ADP}, and effectively establishes that
\eqref{lp:approx-lp} can be solved as the maximization of a weighted norm where
the weights are given by the state-relevance weights. The state-relevance
weights vector does what it's name suggests: impose a balance of the quality of
the approximation of the value function for each state. This means that by
tuning $\vec{c}$ we can give more or less attention to different regions of the
state space $\States$.

Since the state-relevance vector imposes a restriction on how good our
approximations can be. It is natural to question what is the upper bound on the
quality of the approximation. The following theorem establishes bounds in the
quality of approximations. Said theorem corresponds to Theorem 1 in
\Cite[pg.~853]{farias2003LP2ADP}.

\begin{thrm}{}{farias-vanroy-thm1}
    Let $\vec{v}$ such that $\vec{v} \geq T^{*} \vec{v}$, then
    \[
       \left\| \vec{v}_{\pi_v} - \vec{v}_* \right\| \leq \frac{1}{1 - \gamma} \left\| \vec{v}  - \vec{v}_* \right\|_{1, \mu_{\pi_v, \, \sigma}}.
    \]
\end{thrm}

We use the notation $\pi_v$ to mean the policy that yields the value $v$. This
policy can be extracted once a specific $v$ is known. Theorem
\ref{thrm:farias-vanroy-thm1} assures us that if the approximate value function
$\vec{v}$ found by solving the LP is close to the optimal $\vec{v}_{*}$, then
the performance of the policy generated by $\vec{v}, \pi_v$  will also be close
to the perfomance achieved by the optimal policy. By combining theorem
\ref{thrm:farias-vanroy-thm1} with lemma \ref{lem:farias-vanroy-lem1} we
conclude that we would like the initial state-relevance weight vector to capture
the (discounted) frequency with which different states are expected to be
visited. In other words we would like for $\vec{c}$ to be as close as possible
to $\mu_{\pi_v, \, \sigma}$. That is, we would like to invest more effort
approximating the function for the states the learning agent is most likely to
visit, compromising on worse approximations for infrequent states.

\subsection{Error bounds for the ALP}
We begin with a simple bound for the error of the ALP, expressing it as a
function of the minimal error given the selected basis functions.

\begin{thrm}{}{farias-vanroy-thm2}
    Let $\mu: \States \to [0,1]$ be a probability distribution over the state
    space. If $\vec{\beta}$ is an optimal solution to \eqref{lp:approx-lp},
    \[
        \left\| \vec{v}_* - \Phi \vec{\beta} \right\|_{1, \, \mu} \leq \frac{2}{1-\alpha} \min_{\vec{\beta}} \left\| \vec{v}_* - \Phi \vec{\beta} \right\|_{\infty}.
    \]
\end{thrm}

Theorem \ref{thrm:farias-vanroy-thm2} tells us that whenever the optimal value
function lies close to the span of the selected basis functions the ALP
generates a good approximation. If the optimal value function lies within the
span, then the ALP produces an exact solution.

The error bounds we can achieve can be improved much futher by considering Lyapunov functions.

\section{Constraint Sampling}

As previously mentioned, the \eqref{lp:approx-lp} has $|\States| \times
|\Actions|$ constraints, often a prohibitive amount since this number is subject
to the curse of dimensionality. However, since the number of variables is $K \ll
|\States|$, most of the constraints are linearly dependent on the rest and
therefore removing them does not affect the feasible region of the linear
program. This allow for the Approximate Linear Program to be transformed into an
equivalent LP with a much smaller number of constraints by only considering the
linearly independent subset of constraints. We will refer to this new LP as the
RLP (Reduced Linear Program).

\begin{dfn}{Reduced Linear Program (RLP)}{}
    Let $m$ be a constraint sample size such that a LP with $m$ constraints is
    tractable. Let $\psi: \States \times \Actions \to [0, 1]$ be a probability
    distribution over all state-action pairs, and $\mathcal{X}$ a subset of all
    state-action pairs independently sampled according to $\psi$ such that
    $|\mathcal{X}| = m$. Let $\mathcal{N} \subseteq \R^K$ be a subset that
    contains $\Phi\vec{\beta}$.  The \emph{Reduced Linear Program} is defined by
    \begin{equation}
        \label{lp:reduced-lp}
        \tag{RLP}
        \begin{array}{rl@{}ll}
            \displaystyle \min_{\vec{\beta} \in \R^{K}} & \vec{c}^{\top} \Phi \vec{\beta} \\
            \text{S.t.} & \displaystyle \Phi \vec{\beta} \geq \vec{R}_a + \gamma \vec{P}_a \Phi \vec{\beta} & \quad \forall a \in \mathcal{X} \\
            & \vec{\beta} \in \mathcal{N}.
        \end{array}
    \end{equation}
\end{dfn}

\citeauthor{farias2004constraint} focus on two approaches: generating
near-feasible solutions, and bounding the number of constraints to be sampled so
that the RLP generates a solution that closely approximates an optimal solution
to the ALP. We will be considering constraints of the form:
\[
    \alpha_{z}^{\top} \vec{\beta} + k_z \geq 0, \quad \forall z \in Z
\]
on variables $\vec{\beta} \in \R^{K}$, and a set of indices $Z$.

In the case where we want to generate near-feasible solutions, we cannot
guarantee that all constraints will be satisfied over the feasible region. Since
we limit the approximation to a subset of constraints, we consider a subset to
be good if we can guarantee that if we satisfy all constraints in the subset,
the set of unsatisfied constraints has a small measure. The next theorem
(corresponding to theorem 2.1 in \cite[pg.~467]{farias2004constraint})
establishes a bound on the number $m$ of, possible repetated, sampled
constraints necessary to ensure with probability $1 - \delta$ lead to
near-feasibility.

\begin{thrm}{}{fariasvanroy-thm2.1}
    For any $\delta \in (0,1)$, $\varepsilon \in (0,1)$ and $m$ that satisfies
    \[
        m \geq \frac{4}{\varepsilon} \left( K \ln \frac{12}{\varepsilon} + \ln \frac{2}{\delta} \right),
    \]
    a set $\mathcal{X}$ of $m$ identically, independently distributed vandom
    variables drawn from $|\States| \times |\Actions|$ according to
    distribuition $\psi$, satisfies 
    \[
        \IP{
            \sup_{\substack{\vec{\beta} \mid \alpha_{z}^{\top} \vec{\beta} + k_{z} \geq 0  \\ \forall z \in \mathcal{X} }}
            \psi \left( \left\{ y \vertsep \alpha_{y}^{\top} \vec{\beta} + k_y  < 0 \right\} \right) \leq \varepsilon
        } \geq 1 - \delta.
    \]
\end{thrm}

Theorem \ref{thrm:fariasvanroy-thm2.1} provides some reassurance on the error we
could expect to commit even on the worst case sampling scenario. Even without
applying any special knowledge about the problem, we can guarantee with a
probability greater than $1 - \delta$ that we violate restrictions with a
probability smaller than $\varepsilon$.

\section{Closing thoughts \& Bibliographical notes}

In the present chapter we analyzed with greater care the benefits the ALP and
later the RLP would carry in practice in contrast to some of the more
traditional methods for solving RL problems through approximate dynamic
programming. Even though we managed to escape to a certain degree the curse of
dimensionality, we had to accept the cost of an increased number of parameters
(choosing a distribution $\psi$, a number $m$ of constraints, and a subset
$\mathcal{N}$), the need for heuristic approaches to setting parameters,
decreased accuracy, to name the first that come to mind. This is no surprise as
computational mathematics are often govenerned by the ``no free lunch''
theorem\footnote{Not an actual theorem per-se.}, and if we sought more
computationally attainable methods we must accept the trade-off. How much more
efficient an implementation of ALP would be compared to traditional methods for
solving RL problems or even plain simple greedy approaches remains unclear.
Having established the bases, in part \ref{part:III} we hope to answer this
question for a specific problem: the fitting of Classification and Regresssion
Trees. In the next chapter we propose why would framing the CART fitting problem
as an RL problem might be warranted.

\subsection{Bibliographical notes}

This chapter is almost entirely a reproduction of results presented by
\citeauthor{farias2003LP2ADP} in two articles: \citetitle{farias2003LP2ADP} and
\citetitle{farias2004constraint}. The notation is changed to the one introduced
and used throughout this thesis as the original notation is better suited to
areas in queueing theory and inventory management. The results are also given
some context and interpretation based on the themes and goals of this present
thesis. For more detail and proofs on the results presented here please refer to
\cite{farias2003LP2ADP} and \cite{farias2004constraint} respectively.