%===============================================%
% --- Dummy content for test-print purposes --- %
%===============================================%
\part{Primera parte}
\chapter{Hola}

\section{Sección Primera}
\lipsum[1-2]

\subsection{Subsección}
Un poco de matemáticas:

A \emph{policy} is a mapping from states to actions $\pi: S \to A$.

A \emph{value function} from a policy, written $V^{\pi}: S \to \R$ gives the expected sum 
of discounted rewards when acting under that policy
\begin{equation}
	V^{\pi} (s) = \E*{\sum_{t=0}^{\infty} \gamma^t R(s_t) \vertsep s_0 =s, a_t = \pi(s_t), 
	s_{t+1} \mid s_t a_t \sim P}
\end{equation}

Ahora referenciamos a \cite{ths:RodZ} para probar el hightlighting de cita. Ahora pruebo 
los theorem-like environments.

\begin{thrm}{Borel--Carathéodory}{borel-cara}
Let a function $f$ be analytic on a \emph{closed disc} of radius $R$ center origin.
Suppose that $r < R$. Then, we have the following inequality:
\begin{equation}
\|f\|_r \le \frac{2r}{R-r} \sup_{|z| \le R} \operatorname{Re} f(z) + \frac{R+r}{R-r} 
|f(0)|.
\end{equation}
\end{thrm}

Ahora, un corolario del teorema \ref{thm:borel-cara} bien inventado.

\begin{coro}{}{}
	Corolario sin título ni label.
\end{coro}

\subsubsection{Prueba de código}
Finalmente, un poco de código.

\begin{lstlisting}[language=julia, caption=Aplicando algoritmo de cifrado]
# Definiendo sistema de Lorenz
function lorenz!(du, u, p, t, σ=10, β=8//3, ρ=28)
    du[1] = σ * (u[2] - u[1])
    du[2] = u[1] * (ρ - u[3]) - u[2]
    du[3] = u[1] * u[2] - β * u[3]
end
\end{lstlisting}

Ahi va un lema:

\begin{lemma}{}{}
	Lemazo.
	\begin{equation}
		\E*{\sum_{k=0}^{\infty} \vertsep \Pe{A} }
	\end{equation}
\end{lemma}

