\chapter{Reinforcement Learning}
\label{chapter:ReinforcementLearning}

In Chapter~\ref{chapter:Motivation}, we presented an elementary example of a
problem that may be solved via Reinforcement Learning: a robot trying to play a
game called Miniopoly. As simple as that example was, the fundamental ideas will
now allow us to delve properly into the theory and formalize some ideas while
hopefully giving satisfying answers to some questions that the intuitive
treatment might have left open.

\section{The Agent \& the Environment} Recall from
Chapter~\ref{chapter:Motivation} that we referred to the robot player as a
``learning agent''. This agent continuously interacts with the rest of the
game and surveys its current state. As we saw earlier, the robot then selects
an action based on this current game state.

To be precise, say that the game starts at some time $t=0$ and ends at $t=T$. We
discretize this period of time into $t \in \{0, 1, 2, \ldots \}$ At each of
these points $t$ in time, the agent finds itself at some state $S_t \in
\States$, where $\States$ represents the set of all possible states. This
defines a stochastic process: a series of random variables $\{ S_t \}_{t = 0, 1
\ldots}$. Stochastic processes are covered in appendix
\ref{appendix:Stochastic}.

Likewise, at each time step $t$, the agent chooses how to interact with an action
$A_t \in \Actions(s)$, where $\Actions(s)$ is the set of all \textit{available}
actions at the current state $s$. This, too, forms a stochastic process. One time
step in the future $t+1$ the agent receives more feedback from the environment,
the so-called reward signal $R_{t+1} \in \Rewards \subseteq \R$. This process
moves forward from $t$ to $t+1$ and so on until the task is over. This defines a
Markov Decision Process (MDP from now on), which we can reorder as follows:
\begin{equation}
	\label{eq:mdp-succession}
	S_0, A_0, R_1, S_1, A_1, R_2, S_2, A_2, R_3, \ldots
\end{equation}

This ordering makes it easy to see why we discretize time
into steps. This illustrates how at each time step, the agent surveys the state
and, based on it, selects a \textit{feasible} action. The next time step begins
when the robot receives a reward signal from the environment. 

% A key idea is subtly hidden in the notation. Notice how only the set of actions
% depends on a particular $s$. Both the set of states $\States$ and the set of
% rewards $\Rewards$ are, in a certain sense, independent of $s$. Why is it that
% only $\Actions(s)$ is dependent on $s$?

The adjective ``Markov'' in Markov Decision Process is precisely what makes the
set of actions special. As covered in appendix \ref{appendix:Stochastic}, a Markov
process is often used to describe stochastic transitions between a set
of---emphasis on terminology here---\textit{states}. In the example on
Chapter~\ref{chapter:Motivation} the states were the literal squares in the
game, along with the amount of money the player had. The state of the game gave
access to some subset of actions, and the future reward obtained depend on the
action taken. Furthermore, as we discussed, not every state is accessible from
every other state.  Thus, it makes sense that for each state, the set of possible
actions is determined by it. Some transitions are just impossible.

Not only are some transitions impossible at certain states, a given action
chosen by the agent will not always result in the same transition from the state
$s$ to the state $s'$, which might sound obvious as we have already established
that an MDP can best model this entire process. To study the probability
of a certain transition from state $s$ to state $s'$, given that a certain action
$a \in \Actions(s)$ was taken, we introduce the \textit{dynamics} function,
keeping with the notation in \cite{SuttonBarto}.

\begin{dfn}{Dynamics function}{dynamics-func}
	For all $s, s' \in \States, r \in \Rewards, a \in 
	\Actions(s)$, we define the \emph{dynamics function} $p: 
	\States \times \Rewards \times \States \times \Actions(s) 
	\to [0, 1]$ as follows:
	\[
		p(s', r \mid s, a) \coloneqq \IP{S_t = s', R_t = r 
		\vertsep S_{t-1} = s, A_{t-1} = a}.
	\]
\end{dfn}

Notice how the dynamics function is just a joint probability density function
over the space of state-reward pairs ($\States \times \Rewards$). This means we
can treat it as such and get valuable information about, for instance, expected
rewards by calculating:
\begin{equation}
	\label{eq:pre-rewards-func}
	\mathbb{E} \left[ R_t \mid S_{t-1} = s, A_{t-1} = a \right].
\end{equation}
% We will be talking about expected rewards often, so we define the expected
% rewards function to simplify notation further on.
The notation above is cumbersome. We would like to define a shorthand function
that calculates \eqref{eq:pre-rewards-func} taking advantage of the dynamics
function defined earlier.

\begin{dfn}{Expected rewards function}{rewards-func}
	For a state action pair $s, \in \States, a \in \Actions$ we define the
	\emph{expected rewards function} or simply, the \emph{rewards function} $r:
	\States \times \Actions \to \R$ as
	\begin{align*}
		r(s, a) &\coloneqq \sum_{r \in \Rewards} r \sum_{s' \in \States} p(s', r \mid s, a) = \mathbb{E} \left[ R_t \mid S_{t-1} = s, A_{t-1} = a \right], \\
		&= \sum_r p(s' \mid s, a).
	\end{align*}
	The last identity corresponds to marginalizing the transition function.
	Marginalization of a probability distribution is covered in appendix
	\ref{appendix:Stochastic}, Lemma \ref{lem:marginalization}.
\end{dfn}

Keep in mind that both the dynamics and the rewards function are merely tools
for us to formalize the process. For instance, the learning agent has very
little or non-existent knowledge of the dynamics underlying transitions. The
only way it can interact with the environment is via rewards. From the agent's
perspective, the environment is a black box. Each time the agent takes a given
action, the black box responds with a new state and a reward. How or why the new
states and rewards are assigned is unknown to the agent (and sometimes not
apparent to us humans doing the modeling either). Even if the dynamics are
unknown, they are not needed to learn.  Toddlers have no concept of Newton's
third law nor need it to learn which things they can push around and which will
push them.

\section{What I Talk About When I Talk About Learning}

The Agent-Environment
interface we just described is surprisingly helpful in framing how humans master
a given task. Toddlers learn to walk by reinforcement.  They receive a negative
reinforcement: pain whenever they stand up and fall.  For each new attempt, the
toddler will try to fall as little as possible, minimizing the total pain. The
way the robot learning Miniopoly operates is perfectly analogous. It receives
positive rewards when winning and negative ones when losing.  The idea that
learning can be framed as an optimization process is formalized as the
\emph{reward hypothesis}.

\begin{remark}{The reward hypothesis}
	The learning goal can be posited as the maximization of the expected reward
	value of the cumulative sum of the rewards received by the learning agent. 
\end{remark}

The other subtle concept involved in our concept of learning is the
ability to make mistakes and avoid them in the future.

Formally, the learning agent acts according to a given policy, a rule of
correspondence between states and actions. This rule is not necessarily
deterministic. The agent is free to explore and respond to the same state
differently each time it encounters it.

\begin{dfn}{Policy}{policy}
	A \emph{policy} $\pi$ is the probability $\pi: \Actions \to [0,1]$ of taking
	action $a$ from a given state $s$ at some time $t$,
	\[
		\pi(a \mid s) \coloneqq \IP{A_t = a \mid S_t = s}.
	\]
\end{dfn}

Our task is to find increasingly ``better'' policies. Better in the sense of the
reward hypothesis. Achieving a ``good'' policy is exactly like mastering a given
task. Abusing our toddler example one final time, a walking toddler learned to
respond to the feeling of falling forward by taking another step.

Are we done, then? 

We just need to decide what makes one policy better than any other one.

\subsection{The value of policies}

As established earlier by the reward hypothesis, the ultimate goal of our
learning task is to maximize the accumulated rewards received by the agent. The
rewards the agent will perceive depend mainly on two things:
\begin{enumerate}
	\item The starting state,
	\item The policy followed by the agent.
\end{enumerate}

From now on, we talk about the \textit{value} of a given state when following a
particular policy with the help of the value function.

\begin{dfn}{State-Value Function}{value-func}
	The \emph{value} of the state $s$ is the expected value of the
	\underline{discounted} sum of the accumulated rewards from time $t$ forward
	when the agent follows the policy $\pi$,
	\[
		v_{\pi} (s) \coloneqq \mathbb{E}_{\pi} \left[\sum_{k=0}^{\infty} \gamma^{k} R_{t+k+1} \vertsep S_{t} = s\right],
	\]
	for all $s \in \States$. The starting time $t$ is not important to the
	accumulated rewards.
\end{dfn}

The state-value function gives us an idea of what happens when following the
succession described in \eqref{eq:mdp-succession} by taking the actions
$A_{t+k}$ that result in obtaining the rewards $R_{t+k+1}$ being accumulated.
The constant $\gamma \in (0, 1)$ is called the \textit{discount factor}. By
considering the discounted sum of accumulated rewards as the value function, we
are modeling the preference for immediate rather than delayed
rewards\footnote{It is essential to consider the discounted sum since the simple
sum of rewards for an infinite MDP may not be finite.}. 

The notation used in definition \ref{dfn:value-func} is a bit abusive. The
symbol $\mathbb{E}_{\pi}$ means the expected value of the random variable $R$
(the rewards), given that the agent follows the policy $\pi$. Recall that the
random variable $R$ depends on the agent's actions.

Figure \ref{fig:violinplot}, way back in Chapter \ref{chapter:Motivation}, is
a graphical representation of $v_\pi (s)$ for the Miniopoly game for each square
in the board! Actually, it is not \textit{exactly} the value function for the
game Miniopoly, since the states of that game are determined not only by
\textit{where} on the board the player is but also on other factors such as
ownership of the square and how much money is available. The important thing is,
as we showed earlier, the value function can be estimated! This idea is key to
developing the algorithms that allow reinforcement learning to be used in
practice.

For the practical implementations of Reinforcement Learning, it is also handy to
consider how deviating from the chosen policy only at the present time and
follow it for all subsequent decisions affects the accumulated rewards. This can
help us evaluate if the current policy is on track to be the best available or
if we can find an action that would yield better results in the long run. To
evaluate the impacts on the value of a state by choosing action $a$ we introduce
the following function. 

\begin{dfn}{Action-Value Function}{action-value-func}
	We define the value of taking action $a$ in state $s$ at time $t$ and
	following policy $\pi$ for times $t+k$ once again as the discounted sum of
	accumulated rewards:
	\[
		q_\pi (s, a) \coloneqq \mathbb{E}_{\pi} \left[ \sum_{k=0}^{\infty} \gamma^{k} R_{t+k+1} \vertsep S_t = s, A_t = a \right].
	\]
\end{dfn}

One of the characteristic properties of the value functions we have introduced
this chapter, and indeed more generally, the problems that arise from optimal
strategies in MDPs is that they can be written recursively. In other words, to
evaluate the value of a given state; we can ``play it out'' and find that value
as a combination of the values of all the possible states, the agent will visit in
the future.

\begin{lemma}{Recursive evaluation}{recursive-eval}
	The value function $v_\pi$ can be written in terms of itself as
	\[
		v_\pi (s) = \sum_{a \in \Actions} \pi(a \mid s) \sum_{s', r} p(s', r \mid s, a) \left[ r + \gamma \, v_\pi (s') \right].
	\]
\end{lemma}

Lemma \ref{lem:recursive-eval} will be the basis for the methods we will review
further in this thesis. What may not be obvious is that there may exist a
trivially-evaluable value function. The agent's last action can only come from a
relatively small list of possible actions; the computational burden of exploring
said list is often realistic. The second to last action taken comes from a much
more extensive list of possible actions. However, we do not have to explore
every combination since we already have the values computed for each terminal
action.  Using the lemma, we can efficiently compute the value for each
state-action pair for the second-to-last step in the process and so on for the
$n$th-to-last. In simple terms, we can evaluate by ``starting from the end'' and
playing back to the beginning. This relationship was first noted by Bellman
\cite{bellman1957}, one of the pioneers of \textit{dynamic programming}.

\begin{proof}[Proof of Lemma \ref{lem:recursive-eval}]
	We define:
	\[
		G_t \coloneqq \sum_{k=t+1}^{\infty} \gamma^{k-t-1} R_k = R_{t+1} + \gamma R_{t+2} + \gamma^{2} R_{t+3} + \dots, 
	\]
	to make the definition of $v_\pi (s)$ more compact.
	\begin{align}
		v_\pi (s) &= \mathbb{E}_\pi \left[ G_t \vertsep S_t = s \right] \nonumber \\
		&= \mathbb{E}_\pi \left[ R_{t+1} + \gamma G_{t+1} \vertsep S_t = s \right] \nonumber \\
		\label{eq:paso-2}
		&= \mathbb{E}_\pi [R_{t+1} \mid S_t = s] + \gamma \mathbb{E}_\pi \left[ G_{t+1} \vertsep S_{t} = s \right].
	\end{align}

	To calculate the expected value of $R_{t+1}$ with the functions defined
	earlier, we use the following argument. We need to calculate
	\[
		\mathbb{E}_\pi [R_{t+1} \mid S_t = s] = \sum_{r} r \cdot \IP{R_{t+1} = r \mid S_t = s},
	\]
	but the probability on the right hand side is not immediately available. We
	calculate it by using the definition of conditional probability and
	marginalizing probabilities:
	\begin{align}
		p(s', r \mid s, a) &\coloneqq \IP{S_{t+1} = s', R_{t+1} = r \mid S_t = s, A_t = a}, \nonumber \\
		\pi(a \mid s) p(s', r \mid s, a) &= \IP{S_{t+1} = s', R_{t+1} = r, \, A_t = a \mid S_{t} = s}, \nonumber \\
		\sum_{a} \pi(a \mid s) p(s', r \mid s, a) &= \IP{S_{t+1} = s', R_{t+1} = r \mid S_{t} = s}, \nonumber \\
		\sum_{a, s'} \pi(a \mid s) p(s', r \mid s, a) &= \IP{R_{t+1} = r \mid S_{t} = s} , \nonumber\\
		\sum_{a} \pi(a \mid s) \sum_{s'} p(s', r \mid s, a) &=  \IP{R_{t+1} = r \mid S_{t} = s} \nonumber \\
		\sum_{a} \pi(a \mid s) \sum_{s', r} p(s', r \mid s, a) \, r &= \mathbb{E}_\pi \left[ R_{t+1} \vertsep S_t = s \right]. \label{eq:cacho-uno}
	\end{align}

	We can find $\mathbb{E}_\pi [G_{t+1} \mid S_t = s]$ by similar arguments.
	The transition from state $s$ to state $s'$ given that an action $a$ was
	taken happens with probability
	\[
		\sum_a \pi(a \mid s) \sum_{s', r} p(s', r \mid s, a).
	\]
	Therefore,
	\begin{align}
		\label{eq:cacho-dos}
		\mathbb{E}_\pi [G_{t+1} \mid S_t = s] &= \sum_a \pi(a \mid s) p(s', r \mid s, a) \mathbb{E}_\pi \left[ G_{t+1} \mid S_{t+1} = s' \right], \nonumber \\
		&= \sum_a \pi(a \mid s) p(s', r \mid s, a) v_\pi (s').
	\end{align}

	Combining \eqref{eq:cacho-uno} and \eqref{eq:cacho-dos}, making a substitution in \eqref{eq:paso-2} and factoring, we get:
	\begin{equation}
		\label{eq:bellman-equation-v}
		v_\pi (s) = \sum_a \pi(a \mid s) \sum_{s', r} p(s', r \mid s, a) \left[ r + \gamma v_\pi (s') \right].
	\end{equation}
Equation \eqref{eq:bellman-equation-v} is called Bellman's equation for the
value function.
\end{proof}

\section{Striving for the best}

As discussed, we are looking for a policy that achieves the maximum possible value
of the value function of a given state, denoted $v_*(s)$ and defined as
\[
	v_* (s) \coloneqq \max_{\pi} v_\pi (s),
\]
for all $s$ in $\States$. Notice how $v_* \in \R$. Similarly, we can also
define the optimal action-value function, denoted $q_*$, as
\[
	q_* (s, a) \coloneqq \max_{\pi} q_\pi (s, a),
\]
for all $s \in \States, a \in \Actions(s)$. The value $q_*$ represents the
expected return for taking action $a$ in state $s$ at time $t$ and then
following the optimal policy for $t+k$. Formally this can be written as
\[
	q_{*} (s, a) = \mathbb{E} \left[ R_{t+1} + \gamma v_* (S_{t+1}) \vertsep S_t = s, A_t = a \right],
\]
Which can be further simplified by using our previously defined state-value function as:
\begin{equation}
	\label{eq:q-by-v}
	q_\pi (s, a) = r(s, a) + \gamma \sum_{s'} p(s', r \mid s, a) \, v_\pi (s),
\end{equation}
for any policy $\pi$. In particular, for the optimal state-value and action-value functions, we have:
\begin{equation}
	q_*	(s, a) = r(s, a) + \gamma \sum_{s'} p(s', r \mid s, a) \, v_* (s).
\end{equation}

The Recursive Evaluation Lemma \ref{lem:recursive-eval} gives a recursive
relationship any value function must satisfy. However, $v_*$ is special. It can
be written in a special form \cite{bellman1957,SuttonBarto,raoRL4F} without
having to reference any specific policy as:
\begin{equation}
	\label{eq:prop-dependencia-vq}
	v_* (s) = \max_{a \in \Actions(s)} q_* (s, a) \quad \forall s \in \States.
\end{equation}
This relationship is called \textit{Bellman's optimality equation} for the value
function. Intuitively, this relationship is very natural. It states that the
value of a state under an optimal policy must equal the expected return for the
best action from that state \cite[Ch.~3.6]{SuttonBarto}.

Since the state-value and action-value functions are intimately related to one
another we can express the idea behind Bellman's Optimality equation in terms of
the action-value function as well
\[
	q_* (s, a) = \sum_{s', r} p(s', r \mid s, a) \left[ r + \gamma \max_{a'} q_{*} (s', a') \right].
\]

Bellman's optimality equation for the value function is, in fact, a system of
equations, one for each state $s$. The system of equations can be solved
explicitly when considering a deterministic policy, and the number of states is
small. However, it becomes computationally inefficient or simply intractable
when the number of states grows because of the curse of dimensionality, as the
RL problem is combinatoric in nature. Solving the system of equations explicitly
is equivalent to listing out every possible combination of states and actions.

\begin{remark}{The Curse of Dimensionality}
	By ``curse of dimensionality'', we refer to the fact that in combinatorial
	problems such as listing all the action-value pairs of the RL problem,
	whenever the number of states or actions grows linearly, the combinations
	grow exponentially or factorially.
\end{remark}

Other methods to solve Bellman's equations are part of an area called dynamic
programming. Dynamic programming refers to a collection of algorithms and
heuristics that can be used to solve problems with a similar structure to the RL
problem. Dynamic programming (DP) is prevalent in areas such as Operations
Research and other disciplines in which the nature of the problems is episodic
and demands having the ability to adapt \textit{dynamically}. DP problems are
often, as in this case, practically impossible to solve exactly in a
computationally efficient way.

Most of the methods implemented in practical applications to solve MDPs and in
the implementations of Reinforcement Learning are based on techniques that stem
from the application of DP concepts. In contrast, this thesis focuses on a
technique that avoids DP completely to try instead to leverage the properties of
other optimization problems. Namely, the ability to compute efficiently.