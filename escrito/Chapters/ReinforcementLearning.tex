
\label{chapter:ReinforcementLearning}

On Chapter~\ref{chapter:motivation} we presented a very simple 
example of a problem that may be solved via Reinforcement 
Learning. As simple as that example was, the key ideas will now 
allow us to delve into the theory proper, and formalize some 
ideas while hopefully giving satisfying answers to some 
questions that the intuitive treatment might have left open.

\section{The Agent \& the Environment}
Recall from Chapter~\ref{chapter:motivation} that we referred 
to the robot player as a ``learning agent''. This agent is 
continuously interacting with the rest of the game and 
surveying its current state. As we saw earlier, the robot then 
selects an action on the basis of this current game state.

To be precise, say that the game starts at some $t=0$ and ends 
at $t=T$. We discretize this period of time into $t = 0, 1, 2, 
\ldots$ At each of this points $t$ in time, the agent finds 
itself at some state $S_t \in \States$, where $\States$ 
represents the set of all possible states. This defines exactly 
what we called a stochastic process back in 
Chapter~\ref{chapter:stochastic}: a series of random variables 
$\{ S_t \}_{t = 0, 1 \ldots}$. 

Likewise, at each time step $t$ the agent chooses how to 
interact with an action $A_t \in \Actions(s)$, where 
$\Actions(s)$ is the set of all \textit{available} actions at 
the current state $s$. This too forms a stochastic process. One 
time step in the future $t+1$ the agent receives more feedback 
from the environment, the so-called reward signal $R_{t+1} \in 
\Rewards \subseteq \R$. This process moves forward from $t$ to 
$t+1$ and so on until the task is over. This defines a Markov 
Decision Process (MDP from now on), which we can reorder as 
follows:
\begin{equation}
	S_0, A_0, R_1, A_1, R_2, S_2, A_2, R_3, \ldots
\end{equation}

This particular ordering makes it easy to see why we chose to 
discretize time into steps. This illustrates how at each time 
step the agent surveys the state, and based on it selects a 
\textit{feasible} action. The next time step begins when the 
robot receives a reward signal from the environment. 

The key point above is subtly hidden in the notation. Notice 
how only the set of actions depends on a particular $s$. Both 
the set of states $\States$ and the set of rewards $\Rewards$ 
are in a certain sense independent of $s$. Why is it that only 
$\Actions(s)$ is dependent on $s$?

The ``Markov'' in Markov Decision Process is exactly what makes 
the set of actions special. As covered in 
Chapter~\ref{chapter:stochastic}, a Markov process is often 
used to describe stochastic transitions between a set of, 
emphasize on terminology here, \textit{states}. In the example 
on Chapter~\ref{chapter:motivation} the states were the literal 
squares in the game. And as we discussed, not every state is 
accessible from any other state. Thus, it makes sense that for 
each state the set of possible actions is determined by it. 
Some transitions are just impossible.
