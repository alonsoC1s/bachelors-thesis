\documentclass[colorful]{sty/itam-thesis}

\usepackage{lipsum}

\usepackage[
		showframe,
		paperwidth=17cm,
		paperheight=22.5cm,
		nofoot=true,
		bindingoffset=1.1cm,
		inner=1.6cm,
		outer=1.8cm,
		top=2.5cm,
		bottom=1.5cm
	]{geometry}

\usepackage{tikz}
\usepackage{tikzscale}
\usetikzlibrary{backgrounds, intersections, fillbetween}
% \usetikzlibrary{external}
% \tikzexternalize[prefix=tikz/]
\usepackage{import}

% Bibliografía
\usepackage{csquotes}
\usepackage[
	backend=biber,
    citestyle=alphabetic,
    style=alphabetic,
    maxcitenames=2,
    ]{biblatex}
\addbibresource{BachelorsThesis.bib}

% Paquete custom
\usepackage{sty/thesis-package}

\author{Alonso Martinez Cisneros}
\title{Approximate solutions to the Reinforcement Learning 
Problem via Linear Programming; with applications to Supervised 
Machine Learning.}
\date{2022}

\begin{document}

\frontmatter
\pagenumbering{roman}
\maketitle
\makefrontmatter

% Something
\cleardoublepage
% \pagestyle{plain}
% \pagenumbering{arabic}

\addcontentsline{toc}{chapter}{Dedication}

\vspace*{3cm}

\begin{center}
    Por escribir
\end{center}

\medskip
 \cleardoublepage
\addcontentsline{toc}{chapter}{Acknowledgements}

\begin{flushright}{\slshape
    We have seen that computer programming is an art, \\
    because it applies accumulated knowledge to the world, \\
    because it requires skill and ingenuity, and especially \\
    because it produces objects of beauty.} \\ \medskip
    % --- \defcitealias{knuth:1974}{Donald E. Knuth}\citetalias{knuth:1974} \citep{knuth:1974}
\end{flushright}



\bigskip

{
    \chapter*{Acknowledgments}
    Put your acknowledgments here.

    To acknowledge:
    \begin{itemize}
        \item Gente
        \item Open source projects
    \end{itemize}

    \bigskip

    Special thanks Arnaud Joly, core developer emeritus of the scikit-learn
    library, for his assitance in implementing the \ac{cart} algorithm; and Prof.
    Ashwin Rao from Stanford University for pointing me in the right direction.
} \cleardoublepage
\addcontentsline{toc}{chapter}{Abstract}

\chapter*{Abstract}
El resumen chido I guess \cleardoublepage

\tableofcontents
\listofalgorithms
\cleardoublepage

\mainmatter

%====================================================%
% Start of the content proper ---------------------- %
%====================================================%
\pagestyle{plain}
\pagenumbering{arabic}

% %===============================================%
% --- Dummy content for test-print purposes --- %
%===============================================%
\part{Primera parte}
\chapter{Hola}

\section{Sección Primera}
\lipsum[1-2]

\subsection{Subsección}
Un poco de matemáticas:

A \emph{policy} is a mapping from states to actions $\pi: S \to A$.

A \emph{value function} from a policy, written $V^{\pi}: S \to \R$ gives the expected sum 
of discounted rewards when acting under that policy
\begin{equation}
	V^{\pi} (s) = \E*{\sum_{t=0}^{\infty} \gamma^t R(s_t) \vertsep s_0 =s, a_t = \pi(s_t), 
	s_{t+1} \mid s_t a_t \sim P}
\end{equation}

Ahora referenciamos a \cite{ths:RodZ} para probar el hightlighting de cita. Ahora pruebo 
los theorem-like environments.

\begin{thrm}{Borel--Carathéodory}{borel-cara}
Let a function $f$ be analytic on a \emph{closed disc} of radius $R$ center origin.
Suppose that $r < R$. Then, we have the following inequality:
\begin{equation}
\|f\|_r \le \frac{2r}{R-r} \sup_{|z| \le R} \operatorname{Re} f(z) + \frac{R+r}{R-r} 
|f(0)|.
\end{equation}
\end{thrm}

Ahora, un corolario del teorema \ref{thm:borel-cara} bien inventado.

\begin{coro}{}{}
	Corolario sin título ni label.
\end{coro}

\subsubsection{Prueba de código}
Finalmente, un poco de código.

\begin{lstlisting}[language=julia, caption=Aplicando algoritmo de cifrado]
# Definiendo sistema de Lorenz
function lorenz!(du, u, p, t, σ=10, β=8//3, ρ=28)
    du[1] = σ * (u[2] - u[1])
    du[2] = u[1] * (ρ - u[3]) - u[2]
    du[3] = u[1] * u[2] - β * u[3]
end
\end{lstlisting}

Ahi va un lema:

\begin{lemma}{}{}
	Lemazo.
	\begin{equation}
		\E*{\sum_{k=0}^{\infty} \vertsep \Pe{A} }
	\end{equation}
\end{lemma}


\chapter*{Introduction}
\label{chapter:Intro}

In the last 20 or so years, several advancements and novel techniques have
transformed the landscape of the discipline we currently call Artificial
Intelligence. These new approaches have made possible tasks deemed intractable
decades prior, such as natural language processing, image recognition, and
human-level competence at certain games, to name a few. These advancements have
come from a fruitful synergy between several fields of study: Mathematics and
Computer Science, to be precise.

As these techniques have moved from being the \textit{state of the art} to
mainly becoming a problem of engineering, most the people currently implementing
solutions based on Artificial Intelligence today have limited knowledge of the
mathematical underpinnings that enable such powerful methods, for they do not 
need it to do their job. This thesis represents an effort to explore the theory
and intuitions behind one of the most innovative techniques in Artificial
Intelligence: Reinforcement Learning. This work aims to explore the key ideas in
the areas of mathematics that provide the foundations for Reinforcement
Learning: Stochastic Processes, Probability Theory, and a particular emphasis on
Mathematical Optimization, so the field and problems can be presented in an
engaging fashion while sacrificing as little clarity as possible. Then, some 
selected problems and applications will be presented to illustrate the power of
the techniques the reader has become acquainted to. 

As a mathematician, I feel a special obligation to share the beauty of the ideas
in our field with those outside of it. To that end, to make this work more
enjoyable to the people I dedicate it to, I try to emulate a more leisurely
style than the one found in research papers.  Some terseness will be sacrificed
in favor of clarity. Nevertheless, the mathematical minutia will not be ``swept
under the rug'' or separated from the main body of text. Instead, I will try to
give context to the development of mathematical ideas while not insisting on
subjecting the reader to every single technical detail necessary for the
formality of the arguments. The involved details will be available to those
curious in the appendixes.

This thesis consists of mainly three parts, and several chapters that make up
said parts. The first part presents the main ideas from the different fields of
mathematics that will be needed to motivate and justify the theory behind
Reinforcement Learning. The second part explores the Reinforcement Learning
problems as a matter of Mathematics and how these problems are solved through
different optimization techniques. The third and final part deals with
applications of Reinforcement Learning.

Interspersed among the purely mathematical theory are examples of how these
ideas are expressed in the \href{https://julialang.org/}{Julia Programming
Language}. These code snippets will be primarily self-contained, ready to run
and leverage the vast ecosystem that makes Julia ideal for Mathematics and
Scientific disciplines. Some code examples will play the role of pseudo-code as
Julia's simple syntax allows the mathematical ideas to shine through the
implementation details.

\part{Foundations}
\label{part:I}

	\label{chapter:Motivation}
\section{A ``hand-wavy'' first approach}

What we refer to as Artificial Intelligence is a broad 
collection of methods and techniques used to solve a variety
of problems that we collectively associate with ``human 
intelligence'', such as identifying and classifying images, 
processing natural language, and learning from data.  Roughly 
speaking, the discipline can be divided into the following 
sub-fields of research:

\begin{itemize}
	\item Neural Networks.
	\item Computer vision.
	\item Natural Language Processing.
	\item Speech processing.
	\item Machine Learning.
\end{itemize}

This work is concerned specifically with Machine Learning.  
Particularly with a subset of the field known as Reinforcement 
Learning. Some authors such as \cite{SuttonBarto} divide the 
field of Machine Learning (ML) in three parts: supervised 
learning, unsupervised learning, and reinforcement learning.  
Both supervised and unsupervised learning rely heavily on 
statistics-based techniques such as regression models and 
statistical inference. The techniques used in reinforcement 
learning (RL), rely on different models of what learning is, 
and tries to go beyond finding association rules for a simple, 
well-defined problem.

In contrast to supervised learning where a learning agent is 
given data labeled by a knowledgeable source and must ``learn'' 
to classify based on those initial labels, in unsupervised 
learning as well as reinforcement learning, there is no 
``training data'', the agent must act and optimize its strategy 
based only upon the reward or penalty resulting from making a 
certain decision. The key words here are \textit{optimize}, 
\textit{strategy} and \textit{decision}.

For a concrete example picture an android at a casino playing 
at a slot machine, only this machine has $k$ levers instead of 
just one.  This robot is instructed to win as much money as 
possible across some number of games. How could this be 
achieved? Surely each lever makes the machine operate 
differently and thus is more or less likely to get a jackpot. 
If this android were a person, the approach would probably be 
to, across many games, try each lever and see how much money 
comes out. Hopefully, after the first few games this human 
player has a clear idea of which are the best levers to pull. 
This too might be a good strategy for the robot to employ, so 
lets explore that idea further.

If you think about it, the slot machine analogy is not so 
different from the standard mental model we have of say an 
infant learning to crawl.  With poor vision and only its 
caretaker's voice as guide, it must learn to find its way to 
safety by trial and error. It cannot be given millions of 
examples of ``valid paths'' for extrapolation, as is the case 
with supervised learning. This agent learns by interacting with 
the environment itself.

In a certain sense RL leverages our intuitions about the nature 
of learning. All the main elements are there: cause and effect, 
goals, and consequences to decisions made, but as we will see, 
it also encodes more subtle concepts. For instance, delayed 
gratification and planning ahead. It also has the novelty of 
being goal-oriented rather than task-oriented as most Machine 
Learning techniques often are. For example, a self-driving car, 
``trained'' via supervised learning might train on millions of 
examples on what constitutes a valid steering wheel move, is
expected to extrapolate to situations outside the training 
data. Meanwhile a self-driving car, learning by being evaluated 
is learning how to drive as an activity consisting of hundreds 
of little tasks, all to be mastered simultaneously.

\section{Formalizing ideas}
Now, mathematically speaking, what does it \textit{mean} for a 
machine to ``learn''?

We might not get as far as what learning as a whole means for 
humans or machines, but we can certainly discuss what 
mechanisms allow self-driving cars to exist or computers to 
beat world champions of Chess and Go. For that, we need to 
identify the key concepts in the picture I painted and express 
them in the language of mathematics.

For now, we can gloss over the details of how might a machine 
make decisions, perceive goals, take actions or perceive 
rewards. Let us focus on one key component: the ``learning''. 
When we talk about the process of a machine learning something 
we are thinking about something or someone (referred to as 
agent henceforth) that is able to keep track of the decisions 
it made before and whether or not they resulted in positive 
results so it can later on apply that knowledge to become an 
increasingly better problem solver. If this agent is evaluated 
each time it carries out some task, we would like for its score 
to increase each successive time it tries to complete the task. 
The idea of a continuously improving score is the foundation of 
mathematical optimization.

\subsection{Optimization Theory}
As the name suggests, mathematical optimization is concerned 
with finding an ``optimal'' (whatever ``optimal'' means) 
solution to problems where an unambiguous score might be given 
to different solutions. Even if there is no best solution to a 
given problem, the techniques used in mathematical optimization 
often allow for a continuous improvement through iteration.

Going back to our slot machine example, each time the robot 
selects one of the $k$ levers the machine gives out a prize. 
The prizes vary greatly in size, and there is no way to know 
which ones give the biggest prizes in advance. Since the robot 
wants to maximize money earned, if one lever is consistently 
giving out bigger prizes it is likely to keep pulling it. The 
robot was rewarded earlier for choosing that lever by getting a 
big prize.  Similarly if a lever pull resulted in no prize at 
all, the robot lost money by playing and probably will not 
choose that lever again. In each iteration the player robot 
would like to increase more and more the prize, and is thus  
encouraged to keep track of which levers give good prizes, 
improving the average prize in each successive game.

We are now ready to peel away another layer of abstraction.  We 
now have an intuition of what it means for a machine to learn, 
or in other words continually improve. But in the model we 
discussed earlier, the player is making choices along the way. 
A predefined path is not programmed or even known. If it were 
known this would not be a problem at all, the robot could 
choose the known-good levers every single time. Since that is 
not the case the robot must choose and balance between 
exploring and exploiting the knowledge gained so far. What does 
it mean for this robot player to ``make choices''?

\subsection{Stochasticity}
Another key aspect we take for granted when we talk about 
machines learning is implicit in the word learning. If there 
were a predefined path for the robot to take we would hardly 
call that learning, it is merely reproducing instructions.  
What this means for us, is that we must allow for a framework 
in which our robots actions are not completely determined 
beforehand. In fancier words, the succession of events that 
determine the robots actions and responses to stimuli are not 
predetermined or \textit{deterministic}.

This idea is formalized through something called a 
\textit{stochastic process}. The word stochastic is just 
mathematical lingo for ``not entirely determined beforehand'' 
or ``aptly described as a random phenomenon''.  In particular, 
we can think of the robot as being in some state among quite a 
few possible. The robot can change states as time goes by but 
the transitions are not always the same and they do not 
necessarily carry the same reward or punishment.

For instance, the robot at the slot machine might have gotten a 
very big prize the first time it chose a certain lever but got 
almost nothing the next game. The feedback it got by choosing 
an action changes as time goes by, because the slot machine 
behaves randomly. This robot had a certain amount of 
information after pulling the lever for the first time: this is 
a good lever to pull. In the next game this robot already knows 
which lever might be a good bet, so among the $k$ possible 
options it might choose to pull the same lever as last time. 
Next time the reward is much smaller, the robot transitioned 
states in a sense. It was in a state were the best option was 
somewhat clear, to a state in which that one lever is no longer 
the most desirable option. It went from being certain to being 
forced to explore. We might think of this process a navigating 
the branches of a tree, making a decision every time a branch 
forks of into smaller branches.

[AQUI DIAGRAMA DE ARBOL]

\section{A slightly more nuanced example}
One of the major achievements and claims to fame for 
reinforcement learning in the past decades has been its use in 
some of the most famous matches in several classic board games.  
Chess was ``mastered'' when Kasparov, then chess grandmaster, 
was beat by Deep Blue in 1997 \cite{silverchess}, but other 
board games with enormous state spaces such as Go, the game 
that originated in China, proved to be too complex for the 
techniques used by Deep Blue. Deep Blue was mainly based on 
hard-coded openings and endgames taken from other grandmaster 
games, which differs a lot from the RL approach. Go was much 
later attacked by another robot player named AlphaGo, developed 
by researchers from a company named DeepMind 
\cite{silver2017mastering}.  AlphaGo defeated Lee Sedol, the 
reigning human champion in 2016.  AlphaGo was one of the most 
notorious applications of RL and made the field even more 
popular. Since RL has proved to be very good for these kinds of 
tasks in the past, let us explore an example with a simpler 
board game.

\subsection{Duopoly, oligopoly, \ldots}
Consider a simplified version of a popular game whose name I 
avoid using as it might be copyrighted. This game consists of a 
circuit with predefined squares, some available for sale, in 
which a player is at any given moment. In order to move, a 
player spins an arrow fixed on the center of a circle that can 
stop at any of four circle segments numbered from 1 to 4. This 
is the number of squares the player moves forward. You can find 
a scaled version of this game's board in figure 
\ref{fig:miniopoly-board}. The squared numbered 1 through 8 
(colored) are available for purchase, while the white squares 
indicate a special action such as allowing the player to spin 
once more in the same turn, complete the circuit by going 
directly to the starting square and so on.

The robot might acquire that square for itself and if it lands 
on it again, gains some amount of money. The robot also 
receives money every time it completes the circuit. In this 
simplified version the robot is playing alone, but every turn 
some square not currently owned by the robot might become 
unavailable for purchase, and the robot must pay a fee if it 
lands on it from then on. If it lands on jail the robot also 
loses money, a bigger amount. As you might have guessed, the 
point of the game is to make the most amount of money. The game 
ends when the robot goes bankrupt or the amount of squares that 
the robot can buy runs out. What purchase strategy must the 
robot follow to win?
\begin{figure}[H]
	\centering
	\includegraphics[width=.75\textwidth]{img/board.pdf}
	\caption{Board for the game described earlier.}
	\label{fig:miniopoly-board}
\end{figure}

First off, unlike the slot machine-playing robot, this robot 
player cannot entirely choose which square to move to, it can 
only decide if it wants to purchase it or not once it landed on 
it.  For instance, if we look at Fig. \ref{fig:miniopoly-board} 
we see that even if the robot wanted to go to square number 8 
and buy it there is no possible way to do that. The biggest 
number it can spin is 4, and even if it got to roll again the 
furthest it could go is square 5. Let's revise the board and 
simplify by removing the squares that are not accesible on the 
first spin, and lets draw some arrows to make the transitions 
clearer.

\begin{figure}[h]
	\centering
	\includegraphics[width=.75\textwidth]{img/diagram-start.pdf}
	\caption{Squares accessible on the first turn.}
	\label{fig:miniopoly-diagram-start}
\end{figure}

In figure \ref{fig:miniopoly-diagram-start} we can see all the 
possible squares the robot might land on and then buy starting 
from the first turn. Notice how in this figure we only care 
about the starting and ending points, not the transit points, 
so that is why there are no arrows pointing to the squares that 
move the player elsewhere like Spin Again and GoTo Start.
This is the reason there is an arrow that points from ``Start'' 
to itself. This transition might happen. In fact, we know what 
it takes for this to happen: land on ``Roll Again'', and then 
land on ``GoTo Start''. In other words, this transition happens 
if the robot spins a 3 two consecutive times.

% va arriba
Notice how squares 6 and further are not included.
If the robot were to spin a 4, the square with number 3 is the 
furthest it could go. If it were to land on Roll Again and then 
spin a 4, the largest possible spin, it could move four 
squares: from Roll Again to 3, then to 4, then to GoTo Start, 
and finally to square 5, exhausting its possible moves.

If the robot were to land on Roll Again, then spin a 3, it 
would ``land'' on GoTo Start, finishing its turn back at the 
starting square. That particular move results in the robot 
going back to the start and getting paid for completing the 
circuit, without running the risk of landing on a square 
someone owns or jail and getting charged. That sounds like a 
very desirable play (for us looking at the big picture, the 
robot cannot see that far ahead yet). Since we would like this 
to happen, we might ask how likely is for that to happen.

\begin{figure}[h]
	\centering
	\includegraphics[width=\textwidth]{img/transicion.pdf}
	\caption{Transition diagram}
	\label{fig:miniopoly-transicion}
\end{figure}

Lets try to analyze how likely the robot is to end up in this 
particularly rewarding situation. For that it's often helpful 
to think of the possible steps that lead to it. On figure 
\ref{fig:miniopoly-transicion} we can see a more abstract 
diagram of the possible movements the robot might make, and 
this time we are taking into account intermediate squares. Each 
arrow represents where the robot might land after spinning, not 
necessarily where it is when the turn ends. Now, using some 
very basic probability theory we can calculate how likely it is 
for the robot to ``take that route'' so to speak. Once we do, 
we can calculate how likely a ``trajectory'' or sequence of 
arrows is.

% Quitar el p=1 en la flecha de de GS -> S
\begin{figure}[h]
	\centering
	\includegraphics[width=\textwidth]{img/transicion-markov.pdf}
	\caption{Transition diagram with transition probabilities}
	\label{fig:miniopoly-transicion-markov}
\end{figure}

At the moment it is not important why some arrows have certain 
probabilities, and some might not be terribly obvious even for 
someone acquainted with some basic probability. In diagram 
\ref{fig:miniopoly-transicion-markov} to find the probability 
of getting to a certain square (say 6 for example) we look only 
at the last arrow, as it takes into account the probability of 
the last step happening.

The robot we are trying to teach moves as the diagram above 
suggests. For each square there is a different diagram. The 
robot is not aware of this, all it knows is that it spins, and 
then is transported elsewhere and must decide whether to buy or 
not (if the square is available). All it knows is where it has 
been before and if it gained or lost money last turn. 
Crucially, landing on a square previously visited does not 
necessarily mean the reward will be the same as last time. 
Also, the reward for buying a square may come much later on or 
not at all. This student must somehow keep track of what the 
expected reward for landing somewhere might be, recording it 
somewhere and tallying up as it goes. For instance after a few 
turns it might have something fig. \ref{fig:hist-miniopoly} on 
its head to help make decisions.

% Aqui no va el square 6 porque es first turn
\begin{figure}
\centering
\includegraphics[width=\textwidth]{img/hist-miniopoly.tikz}
\label{fig:hist-miniopoly}
\caption{Boxplot of rewards obtained by visiting each square}
\end{figure}

% EXPLICAR MUCHO MEJOR
% SOLO ES UN FIRST TURN SIMULADO MUCHAS VECES CON SQUARES 
% COMPRADOS YA
Its stands to reason that as this histogram is updated the 
learning agent will be able to make progressively better 
choices. Simple as this example might be, the essence is 
similar to the powerful programs that beat world champions.

This example covers most of the core ideas of what is referred 
to as the Reinforcement Learning Problem. We will explore the 
ideas presented here in much more detail along this theses and 
crucially, focus on something we left out on this example. How 
can a purchase strategy be improved upon?

\section{Wrapping up}
So far we have developed a mental model of what to do should we 
wish to teach a robot how to drive or beat a world champion of 
Go. The rest of this thesis is dedicated to the careful 
development of the ideas here presented into the language of 
mathematics.  But beyond mere description, this exercise has 
the potential to unlock \textit{insight}. As is often said by 
legendary math communicator Grant Sanderson\footnote{From the 
YouTube channel 
\href{https://www.youtube.com/channel/UCYO_jab_esuFRV4b17AJtAw}{3blue1brown}}(loose 
quote), the point in formulating things this way is to gain a 
deeper understanding of the phenomenon. So let's dive right in.

	
\label{chapter:Stochastic}

To be filled\ldots

	\chapter{Optimization Theory}
\label{chapter:OptimizationTheory}
	\chapter{Supervised Machine Learning}
\label{chapter:SupervisedLearning}

In Chapter \ref{chapter:Motivation}, we briefly reviewed the areas that
constitute the field of \ac{ml} and motivated an introductory example to
Reinforcement Learning. We highlighted the differences between Supervised
\ac{ml} (supervised \ac{ml}) and \ac{rl} by contrasting the way each work.
Supervised \ac{ml} receives a wealth of examples from which it must extract
patterns, while \ac{rl} receives no input data and so has to learn by
interacting with its environment. Statistically speaking, supervised \ac{ml} is
often tasked with classifying or predicting a response. In this thesis, we focus
only on the classification task.

\section{Classification}

Supervised \ac{ml} algorithms tasked with classifying receive large amounts of
labeled data. Through the process called ``fitting'' or ``learning'', the
algorithm will (hopefully) label correctly new observations never seen before.
The classification task seeks to find a systematic way of predicting a
phenomenon given a set of measurements.

\subsection{A formal description} \label{sss:formalizing-trees}

Putting some concepts to work, picture a scenario where we have access to
several measurements (e.g., age, weight, blood pressure, etc\dots) for a group
of patients. Some patients in the study group suffered a stroke; the rest did
not. Our task is to find a systematic way of predicting whether or not a new
patient, not seen before, will suffer from a stroke by measuring the same
variables measured for the initial group.

Fundamentally, a supervised \ac{ml} algorithm must receive a \textit{training
set}, a set of pre-labeled data from a knowledgeable source, in contrast with
\ac{rl}, where labeling is often not even possible or practical. This source of
truth, the training set, is a set of observations $(\vec{x}_1, y_1), \dots,
(\vec{x}_n, y_n)$. The vectors $\vec{x}_i = [x_{1, i}, x_{2, i}, \dots, x_{p,
i}]^{\top} \in \R^{p}$ can be thought of as a list of measurements of every
variable of interest for the $i$-th observation. In the case of the example
trying to predict strokes, $x_{1, i}$ would correspond to age and $x_{2, i}$ to
weight for the individual corresponding to observation $i$. The training set
also contains $y_1, \dots, y_n$, one-dimensional values we call responses or
labels in the specific case of classification. Using the standard parlance,
following \cite{louppe2014}, the input variables are known as \textit{features},
input vectors as \textit{instances} or \textit{samples} and the output variable
as \textit{target}. For our purposes, we consider that the target variable is
always categoric, not numeric nor continuous. Nevertheless, continuous target
variables are allowed, but the learning task is called regression, which is
outside this thesis's scope.

In a typical \ac{ml} workflow, the algorithm used to make predictions is trained
on the set we just described. Its performance is tested on a different set of
similar data that the algorithm had no access to during its training period.
From now on, we denote the training set as $\L$. For convenience we group the
feature vectors into a \textit{feature matrix} $X \in \R^{p \times n}$, where
the $i$-th column is $\vec{x}_i$. Similarly, we group the target variables into
the vector $\vec{y} \in \R^{n}$.

Our classification task can be framed as finding a function $f_{\L}$ we will
call model whose output or predictions $f_{\L} (\vec{x}) = \widehat{y}$
are ``as good as possible''. We subscript the function $f$ with $\L$ to
highlight that the function depends on what data the training set contains. We
proceed to define what makes a model ``good'' at making predictions.

\subsection{Evaluation}

As it has become a recurring theme in this thesis, determining what makes a
prediction suitable and finding the best possible model $f_\L$ is a process of
optimization. Since we assume that the training set is a sample of the
population we aim to label with our model, it stands to reason that a model that
makes the fewest mistakes when tasked with classifying the training observations
will also make the fewest mistakes when classifying new observations. In
reality, this is not precisely the case, as often making the number of
classification mistakes as small as possible in the training set results in a
model that ``memorized'' the set and has no ability whatsoever to generalize
patterns. It only reproduces known-good answers. But the idea is not entirely
misguided, it provides a good starting point.

The process we referred to as ``fitting'' a model consists of finding a model
which minimizes its expected prediction error, also called generalization error.
We hope that by minimizing this error we can extract as much information from
the training set without hard-coding the labels for future observations.

\begin{dfn}{Prediction or Generalization Error}{generalization-error}
    The expected \emph{prediction} or \emph{generalization} error of a model
    $f_\L$ is the probability of misclassification of the model
    \[
        \Err (f_\L) = \mathbb{E} \left[ \1 (y_i \neq \widehat{y}_i)  \right].
    \]
    Where $\widehat{y}_i \coloneqq f_\L (\vec{x}_i)$ is the model's prediction
    for the observation $\vec{x}_i$, and the indicator function $\1$ equals one
    whenever the condition inside it holds, and zero otherwise.
\end{dfn}

Minimizing this Generalization Error will allow us to classify the newest
observations committing the lowest possible number of errors overall.
Including observations never seen before, not only the ones we have access to in
the training set. Since the distributions for observations $(\vec{x}_j, y_j)$
are generally not known, we must estimate the generalization error. Several
techniques to solve this problem exist but are numerous and outside of scope.
From now on, we denote by $\widehat{\Err}$ an estimator of the generalization
error.

The term ``fitting'' when discussing finding a suitable model is a consequence
of how such a model is found. Since the generalization error is not directly
measurable and thus not directly minimizable, we have to use a reasonable
approximation. We assume that a family of candidate models, $\mathcal{H}$ known
as hypotheses, exists. Our optimization target then becomes to find the best
model among the space of hypotheses. To be more specific, let $\vec{\theta}$ be
the vector of hyper-parameters controlling the behavior of a specific model in
$\mathcal{H}$.  Then, our optimization task is to find $\theta^{*}$,
\[
    \theta^{*} \coloneqq \argmin_{\theta} \widehat{\Err}\left(f_\L (\vec{x}; \theta) \right) \quad \text{ for all } \vec{x} \in \mathcal{X}.
\]
where the vectors of features come from the input space $\mathcal{X}$ defined
clearly in the next section.

Many classification models are available, each leveraging different properties
and resulting in different strengths and weaknesses. The optimization problem
involved in fitting each one may be different from the rest. For this thesis, we
focus on one specific model called classification trees. We introduce them here
and show in part \ref{part:II} how fitting them gives way to an optimization
process analogous to a Reinforcement Learning problem and frame it as such so we
can leverage the power of \ac{rl} to fit classification trees.

\section{Classification Trees}

Classification trees and, more generally, classification and regression trees
are \ac{ml} models like the ones described in the previous section.  They have
gained massive popularity in the recent and ongoing boom in machine learning for
their numerous qualities. For instance, they can model arbitrarily complex
relationships in data, handle both numeric and categorical data, and are easily
interpretable as they result in simple decision rules. Most importantly, they
are the building block of state-of-the-art algorithms for \ac{ml}, such as
XGBoost \cite{XGBoost}.

Sadly, their fitting process leads to a rather complicated problem that cannot
be solved exactly in a reasonable time. The fitting process shares many
similarities with the \ac{rl} problem, which is described in detail in Chapter
\ref{chapter:ReinforcementLearning}. This similarity is the foundational idea of
this thesis, and we hope to show in part \ref{part:II} how an alternative
methodology to fitting trees can be developed by leveraging the techniques used
in \ac{rl}. With that goal in mind, it is time to lay the foundations behind how
trees are structured.

\subsection{Tree models}

Let $\Omega = \left\{ (\vec{x}_i, y_i) \right\}$ be the space of all possible
feature-target pairs. When each feature $y$ is part of a set of categories
$\mathcal{C} \coloneqq \left\{ c_1, c_2, \dots, c_j \right\}$, another way to
look at the classification task is to define a partition over $\Omega$ taking
advantage of the natural distinction our set of categories provides. That
partition can be described as:
\[
    \Omega = \bigcup_{i=1}^{j} \Omega_{c_i},
\]
where each $\Omega_{c_i}$ is defined as $\left\{ (\vec{x}_i, y_i) \mid y_i = c_k
\right\}$.

Similarly, a model $f_\L$ defines a partition. This partition however is made
over the input space $\mathcal{X} = \left\{ \vec{x}_i \mid (\vec{x}_i, y_i) \in
\Omega \right\}$. This partition can be described as the preimages of $f_\L$ as
such
\[
    \mathcal{X} = \bigcup_{i=1}^{j} f_{\L}^{-1}(\Omega_{c_i}).
\]
The classification task then can be thought of as fitting the model $f_\L$ that
gives the partition of $\mathcal{X}$ that most closely approximates the
partition on $\Omega$ as a result of its preimages.

In other words, we are trying to represent $f_\L$ as a tree (in the same way
Computer Science thinks about trees) where any node $t$ represents a
\underline{subset} $\mathcal{X}_t \subseteq \mathcal{X}$ of the input space such
that the node designated as root (denoted $t_0$) corresponds to the entirety of
$\mathcal{X}$.  Internal nodes $t$ are originated via a \textit{split} $s_t$
taken from a set of questions $\mathcal{Q}$.

The set of questions $\mathcal{Q}$ is exactly what it sounds like. The question
defining split $s_t$ might be, for example, is $x_{m, 1} < 65$? Or, is the person
recorded as observation $m$ younger than 65? The space $\mathcal{X}_t$
represented by node $t$ is made up of disjoint subsets corresponding to each
of $t$'s children nodes; two in the case of binary trees. 

Terminal nodes (or leaves) are labeled with a best guess value $\widehat{y}_t
\in \mathcal{C}$. The prediction process is carried out by navigating the tree,
providing answers to the questions defining each node until a leaf is reached.
The label for that leaf will be the tree's prediction. In algorithm
\ref{alg:tree-predict}, we show a formal description of the process of using an
already fitted tree $f_\L$ to predict the label of a new observation. 

\begin{algorithm}
    \SetKwFunction{Predict}{predict}
    \KwIn{The tree $f_\L$}
    \KwIn{The feature vector $\vec{x}$}
    \KwOut{The predicted class $\widehat{y}$ for $\vec{x}$}
    \Function{\Predict{$f_\L, \vec{x}$}}{
        $t \gets t_0$ \;
        \While{$t$ is not a terminal node}{
            $q \gets$ the question in $\mathcal{Q}$ that separates the node $t$. \;
            $t \gets$ the child node $t'$ that corresponds to evaluating $\vec{x}$ at question $q$  \;
        }
        \Return{$\widehat{y}_t$ the value at node $t$} \;
    }
    \caption[Tree prediction algorithm]{Predict output value $\widehat{y}$ with
        tree $f_\L$ \cite[Ch.~3.2]{louppe2014}.}
    \label{alg:tree-predict}
\end{algorithm}

[Maybe una imágen de muestra de el árbol particionando el espacio.]

\subsection{Fitting Decision Trees}

\citeauthor{hyafil1976} proved that fitting a decision tree that minimizes the
generalization error is an NP-complete problem \cite{hyafil1976}. That means no
polynomial time algorithm to solve it exists. As with other NP-complete
problems, the solution methods must rely on heuristics. One of the most widely
used algorithms for tree fitting is the \acl{cart} algorithm developed by
\citeauthor{breiman2017} \cite{breiman2017}; it is implemented in
scikit-learn\footnote{The industry-standard python library for machine learning
\cite{louppe2014}.}, available to R users via the tidymodels extension and part
of MLJ.jl\footnote{The most widely used machine learning library for Julia
users.}. This thesis will focus on \ac{cart} and later on try to extend it.

The \ac{cart} algorithm is based on greedily finding splits based on node
purity, the amount of misclassified observations in a given node. The
\textit{purer} the node, the lower the impurity score, the better the
prediction. We denote by $\L_t$
\[
    \L_t \coloneqq \left\{ (\vec{x}, y) \mid \vec{x} \in \mathcal{X}_t \right\}.
\]

\begin{remark}{Greedy}
    When using the term greedy, we refer to a heuristic used in optimization
    algorithms. The heuristic consists of evaluating the immediately available
    actions without looking ahead to analyze if some action ignores the path
    that will actually lead to the best results down the line. Greedy algorithms
    are used often since looking ahead is often very computationally taxing.
\end{remark}

Many impurity functions can be used to determine the goodness of a split, each
with their benefits and drawbacks. We only discuss the impurity decrease at the
moment, since it is not necessary to have a specific impurity function to
describe the fitting algorithm nor do we have enough space to cover the
available options and their properties.

\begin{dfn}{Impurity decrease}{impurity-decrease}
    The \emph{impurity decrease}, under an impurity function $i: \mathcal{Q} \to
    [0, 1]$, of a binary split of node $t$ into a left child $t_L$ and a right
    child $t_R$ is denoted by $\Delta(i)$, and defined as: 
    \[
        \Delta i(s, t) \coloneqq i(t) - \frac{N_{t_L}}{|\L_t|} \, i(t_L) - \frac{N_{t_R}}{|\L_t|} \, i(t_R),
    \]
    where $N_{t_L}$ and $N_{t_R}$ are the number of observations contained by
    nodes $t_L$ and $t_R$, respectively. The impurity decrease is defined as a
    function of both $t$ and $s$ since the split $s \in \mathcal{Q}$ is what
    defines the children of $t$: $t_R$ and $t_L$.
\end{dfn}

It is straightforward to see the similarity of the impurity decrease definition
with how an expected value is estimated since $N_{t_L} / |\L_t|$ is the
probability of a given observation falling to the left node. In simple words,
the impurity decrease for node $t$ is simply the original impurity minus the
expected value of the impurity of its children.

We have the tools to define a formal procedure for greedily fitting a binary
classification tree in algorithm \ref{alg:tree-fit}. The pseudocode presented
here lacks a litany of essential considerations, such as stopping criteria,
splitting rules, and impurity functions. We are limited to presenting a broad
picture, as exploring even some of the deficiencies mentioned would require
entire books.

\begin{algorithm}
    \SetKwFunction{Fit}{fit}
    \SetKwFunction{Pop}{pop}
    \SetKwFunction{Push}{push}
    \KwIn{The training set $\L$}
    \KwOut{A fitted binary classification tree $f_\L$}
    \Function{\Fit{$f_\L$}}{
        Create an empty tree $f_\L$ with root node $t_0$ \;
        Create an empty stack $S$ of \emph{open} nodes $(t, \L_t)$ \;
        Append $(t_0, \L)$ to $S$ \;
        \While{$S$ is not empty}{
            $t, \, \L_t \gets S$.\Pop{} \;
            \uIf{stopping criteria are met for $t$}{
                Set $\widehat{y}_{t} \gets$ a constant value for node $t$ \;
            }\Else{
                Find the split $s_*$ on $\L_t$ that maximizes impurity decrease
                \[
                    s_* = \argmax_{s \in \mathcal{Q}} \Delta i(s, t)
                \] \;
                Partition $\L_t$ into $\L_{t_L}$ and $\L_{t_R}$ according to $s_*$ \;
                Create the left and right children nodes $t_L, t_R$ of t \;
                Append node $t$ defined by split $s_*$ to the tree $f_\L$ \;
                $S.$\Push{($t_R, \L_{t_R}$)} \;
                $S.$\Push{($t_L, \L_{t_L}$)} \;
            }
        }
        \Return{$f_\L$} \;
    }
    \caption[Greedy tree fitting algorithm]{Greedy fit of a binary
        classification tree \cite[Ch.~3.3]{louppe2014}.}
    \label{alg:tree-fit}
\end{algorithm}

Splitting rules will be crucial for developing the rest of this thesis, so we
must discuss them, even if only briefly. The true magic of the algorithm lies in
the splitting strategies and the termination criteria.

\subsection{Splitting rules}

\begin{dfn}{Split}{split}
    A \emph{split} $s$ of node $t$ is a partition of $\mathcal{X}_t$. Recall
    that the subset $\mathcal{X}_t$ is what we call a node. Formally, a
    partition is a collection of non-empty subsets of $\mathcal{X}_t$ such that,
    \begin{enumerate}
        \item The union of all subsets equals $\mathcal{X}_t$.
        \item The intersection of any two distinct subsets is empty.
    \end{enumerate}
\end{dfn}

The number of possible partitions of $\L_t$, with $N_t$ elements, into $k$
non-empty subsets is given by the Stirling number of the second kind
\cite{louppe2014}
\[
    S(N_t, k) \coloneqq \frac{1}{k!} \sum_{j=0}^{k} (-1)^{k-j} \binom{k}{j} j^{N_t}.
\]
Even in the case of binary partitions in which $S(N_t, k)$ reduces to $2^{N_t
-1}-1$, the number of partitions grows exponentially for the binary case. For
the non-binary case, even worse, it grows factorially. Calculating each possible
partition and choosing the best is not computationally feasible. The attentive
reader will find the similarities when describing in full detail the \ac{rl}
problem presented in Chapter \ref{chapter:ReinforcementLearning}. The best
binary split $s_*$ can be found using algorithm
\ref{alg:best-greedy-binary-split}.

\begin{algorithm}
    \SetKwFunction{GreedyBestSplit}{FindBestGreedyBinarySplit}
    \KwIn{A node $t$ (a subset $\L_t$ of the training set $\L$)}
    \KwOut{The best binary, greedy, split $s_*$ on $\L_t$}
    \Function{\GreedyBestSplit{$t$}}{
        $\Delta \gets - \infty$ \;
        \ForEach{feature $X_j$ with $j \in  \{1, \dots, p\}$}{
            Find the best greedy binary split $s_{*}^{j} \in \mathcal{Q}$ \label{line:best-greedy-split} \;
            \If{$\Delta i(s_{*}^{j}, t) > \Delta$}{
                $\Delta \gets \Delta i(s_{*}^{j}, t)$ \;
                $s_* \gets s_{*}^{j}$ \;
            }
        }
        \Return{$s_{*}$} \;
    }
    \caption[Best binary, greedy, split for node $t$.]{Best binary greedy split $s_*$ for node $t$ \cite[Ch.~3.6.3]{louppe2014}.}
    \label{alg:best-greedy-binary-split}
\end{algorithm}

In simple words, algorithm \ref{alg:best-greedy-binary-split} receives a node
$t$, which is nothing more than a subset of the training set. Then, it iterates
over the features of the data set, and finds the one that woyld yield a bigger
impurity decrease when split. In line \ref{line:best-greedy-split} of the
algorithm, the split $s_{*}^{j}$ is a question pertaining to the $j$-th feature.
If the $j$-th feature of the data set corresponded to the subject's age, one
suitable split would be ``is the subject older than 65?''. This question divides
the observations of the feature $X_j$ in two groups: those older and those
younger than 65. Once those two groups are established, we can calculate the
impurity decrease as the impurity of the current node, minus the weighted sum of
the impurities of the groups of individuals younger and older than 65.
Definition \ref{dfn:impurity-decrease} formalizes the process we just described.

It can be proved that the complexity of algorithm
\ref{alg:best-greedy-binary-split} is constrained by the complexity of the sort
operation used to order the values of $\Delta$ and retrieve the largest
\cite[Ch.~5]{louppe2014}.

With algorithm \ref{alg:best-greedy-binary-split}, we have enough background to
examine the similarities between the problem that arises from fitting a
classification tree and the more general reinforcement learning problem in part
\ref{part:II}.

\section{Bibliographical notes}
As mentioned, we had to skip several details necessary for an actual
implementation of the tree-fitting algorithm. They are not strictly mandatory
for the purposes of this thesis but are vastly interesting and worthwhile in
their own right. To clarify or expand upon the concepts presented here, please
refer to the sources used to develop this chapter:
\begin{enumerate}
    \item \citeauthor{breiman2017}'s book \citetitle{breiman2017}. One of the
        cornerstone texts for this subject and the first formal descriptions of
        the \ac{cart} algorithm \cite{breiman2017}.
    \item The standard textbook for modern machine learning algorithms and
        applications: \citetitle{elements2009}. This source is particularly good
        for reviewing the foundations of statistical, supervised learning in
        much more depth and in a better style that could be achieved here
        \cite{elements2009}.
    \item \citeauthor{louppe2014}'s doctoral thesis, which is heavily cited in
        this chapter. Louppe is a core developer emeritus of the scikit-learn
        python library, the industry-standard library. His work is admirably
        deep, clear, and has valuable insights on the real-world considerations
        a practical implementation would have to keep in mind \cite{louppe2014}.
        After extensive research, this source provided the most satisfying
        treatment of classification and regression trees of all documents
        reviewed.
\end{enumerate}

The field of machine learning is so vast that we only had a chance to develop
the most rudimentary, essential ideas we will need. Reading the sources cited
for this chapter is encouraged.
	\chapter{Reinforcement Learning}
\label{chapter:ReinforcementLearning}

In Chapter~\ref{chapter:Motivation}, we presented an elementary example of a
problem that may be solved via Reinforcement Learning: a robot trying to play a
game called Miniopoly. As simple as that example was, the fundamental ideas will
now allow us to delve properly into the theory and formalize some ideas while
hopefully giving satisfying answers to some questions that the intuitive
treatment might have left open.

\section{The Agent \& the Environment}

Recall from Chapter~\ref{chapter:Motivation} that we referred to the robot
player as a ``learning agent''. This agent continuously interacts with the rest
of the game and surveys its current state. As we saw earlier, the robot then
selects an action based on this current game state.

To be precise, say that the game starts at some time $t=0$ and ends at $t=T$. We
discretize this period of time into $t \in \{0, 1, 2, \ldots \}$ At each of
these points $t$ in time, the agent finds itself at some state $S_t \in
\States$, where $\States$ represents the set of all possible states. This
defines a stochastic process: a series of random variables $\{ S_t \}_{t = 0, 1
\ldots}$. Stochastic processes are covered in appendix
\ref{appendix:Stochastic}.

Likewise, at each time step $t$, the agent chooses how to interact with an action
$A_t \in \Actions(s)$, where $\Actions(s)$ is the set of all \textit{available}
actions at the current state $s$. This, too, forms a stochastic process. One time
step in the future $t+1$ the agent receives more feedback from the environment,
the so-called reward signal $R_{t+1} \in \Rewards \subseteq \R$. This process
moves forward from $t$ to $t+1$ and so on until the task is over. This defines a
Markov Decision Process (\acs{mdp} from now on), which we can reorder as follows:
\begin{equation}
	\label{eq:mdp-succession}
	S_0, A_0, R_1, S_1, A_1, R_2, S_2, A_2, R_3, \ldots
\end{equation}

This ordering makes it easy to see why we discretize time
into steps. This illustrates how at each time step, the agent surveys the state
and, based on it, selects a \textit{feasible} action. The next time step begins
when the robot receives a reward signal from the environment. 

% A key idea is subtly hidden in the notation. Notice how only the set of actions
% depends on a particular $s$. Both the set of states $\States$ and the set of
% rewards $\Rewards$ are, in a certain sense, independent of $s$. Why is it that
% only $\Actions(s)$ is dependent on $s$?

The adjective ``Markov'' in Markov Decision Process is precisely what makes the
set of actions special. As covered in appendix \ref{appendix:Stochastic}, a Markov
process is often used to describe stochastic transitions between a set
of---emphasis on terminology here---\textit{states}. In the example on
Chapter~\ref{chapter:Motivation} the states were the literal squares in the
game, along with the amount of money the player had. The state of the game gave
access to some subset of actions, and the future reward obtained depend on the
action taken. Furthermore, as we discussed, not every state is accessible from
every other state.  Thus, it makes sense that for each state, the set of possible
actions is determined by it. Some transitions are just impossible.

Not only are some transitions impossible at certain states, a given action
chosen by the agent will not always result in the same transition from the state
$s$ to the state $s'$, which might sound obvious as we have already established
that an \ac{mdp} can best model this entire process. To study the probability
of a certain transition from state $s$ to state $s'$, given that a certain action
$a \in \Actions(s)$ was taken, we introduce the \textit{dynamics} function,
keeping with the notation in \cite{SuttonBarto}.

\begin{dfn}{Dynamics function}{dynamics-func}
	For all $s, s' \in \States, r \in \Rewards, a \in 
	\Actions(s)$, we define the \emph{dynamics function} $p: 
	\States \times \Rewards \times \States \times \Actions(s) 
	\to [0, 1]$ as follows:
	\[
		p(s', r \mid s, a) \coloneqq \IP{S_t = s', R_t = r 
		\vertsep S_{t-1} = s, A_{t-1} = a}.
	\]
\end{dfn}

Notice how the dynamics function is just a joint probability density function
over the space of state-reward pairs ($\States \times \Rewards$). This means we
can treat it as such and get valuable information about, for instance, expected
rewards by calculating:
\begin{equation}
	\label{eq:pre-rewards-func}
	\mathbb{E} \left[ R_t \mid S_{t-1} = s, A_{t-1} = a \right].
\end{equation}
% We will be talking about expected rewards often, so we define the expected
% rewards function to simplify notation further on.
The notation above is cumbersome. We would like to define a shorthand function
that calculates \eqref{eq:pre-rewards-func} taking advantage of the dynamics
function defined earlier.

\begin{dfn}{Expected rewards function}{rewards-func}
	For a state action pair $s, \in \States, a \in \Actions$ we define the
	\emph{expected rewards function} or simply, the \emph{rewards function} $r:
	\States \times \Actions \to \R$ as
	\begin{align*}
		r(s, a) &\coloneqq \sum_{r \in \Rewards} r \sum_{s' \in \States} p(s', r \mid s, a) = \mathbb{E} \left[ R_t \mid S_{t-1} = s, A_{t-1} = a \right], \\
		&= \sum_r p(s' \mid s, a).
	\end{align*}
	The last identity corresponds to marginalizing the transition function.
	Marginalization of a probability distribution is covered in appendix
	\ref{appendix:Stochastic}, Lemma \ref{lem:marginalization}.
\end{dfn}

Keep in mind that both the dynamics and the rewards function are merely tools
for us to formalize the process. For instance, the learning agent has very
little or non-existent knowledge of the dynamics underlying transitions. The
only way it can interact with the environment is via rewards. From the agent's
perspective, the environment is a black box. Each time the agent takes a given
action, the black box responds with a new state and a reward. How or why the new
states and rewards are assigned is unknown to the agent (and sometimes not
apparent to us humans doing the modeling either). Even if the dynamics are
unknown, they are not needed to learn.  Toddlers have no concept of Newton's
third law nor need it to learn which things they can push around and which will
push them.

\section{What I Talk About When I Talk About Learning}

The Agent-Environment
interface we just described is surprisingly helpful in framing how humans master
a given task. Toddlers learn to walk by reinforcement.  They receive a negative
reinforcement: pain whenever they stand up and fall.  For each new attempt, the
toddler will try to fall as little as possible, minimizing the total pain. The
way the robot learning Miniopoly operates is perfectly analogous. It receives
positive rewards when winning and negative ones when losing.  The idea that
learning can be framed as an optimization process is formalized as the
\emph{reward hypothesis}.

\begin{remark}{The reward hypothesis}
	The learning goal can be posited as the maximization of the expected reward
	value of the cumulative sum of the rewards received by the learning agent. 
\end{remark}

The other subtle concept involved in our concept of learning is the
ability to make mistakes and avoid them in the future.

Formally, the learning agent acts according to a given policy, a rule of
correspondence between states and actions. This rule is not necessarily
deterministic. The agent is free to explore and respond to the same state
differently each time it encounters it.

\begin{dfn}{Policy}{policy}
	A \emph{policy} $\pi$ is the probability $\pi: \Actions \to [0,1]$ of taking
	action $a$ from a given state $s$ at some time $t$,
	\[
		\pi(a \mid s) \coloneqq \IP{A_t = a \mid S_t = s}.
	\]
\end{dfn}

Our task is to find increasingly ``better'' policies. Better in the sense of the
reward hypothesis. Achieving a ``good'' policy is exactly like mastering a given
task. Abusing our toddler example one final time, a walking toddler learned to
respond to the feeling of falling forward by taking another step.

Are we done, then? 

We just need to decide what makes one policy better than any other one.

\subsection{The value of policies}

As established earlier by the reward hypothesis, the ultimate goal of our
learning task is to maximize the accumulated rewards received by the agent. The
rewards the agent will perceive depend mainly on two things:
\begin{enumerate}
	\item The starting state,
	\item The policy followed by the agent.
\end{enumerate}

From now on, we talk about the \textit{value} of a given state when following a
particular policy with the help of the value function.

\begin{dfn}{State-Value Function}{value-func}
	The \emph{value} of the state $s$ is the expected value of the
	\underline{discounted} sum of the accumulated rewards from time $t$ forward
	when the agent follows the policy $\pi$,
	\[
		v_{\pi} (s) \coloneqq \mathbb{E}_{\pi} \left[\sum_{k=0}^{\infty} \gamma^{k} R_{t+k+1} \vertsep S_{t} = s\right],
	\]
	for all $s \in \States$. The starting time $t$ is not important to the
	accumulated rewards.
\end{dfn}

The state-value function gives us an idea of what happens when following the
succession described in \eqref{eq:mdp-succession} by taking the actions
$A_{t+k}$ that result in obtaining the rewards $R_{t+k+1}$ being accumulated.
The constant $\gamma \in (0, 1)$ is called the \textit{discount factor}. By
considering the discounted sum of accumulated rewards as the value function, we
are modeling the preference for immediate rather than delayed
rewards\footnote{It is essential to consider the discounted sum since the simple
sum of rewards for an infinite \ac{mdp} may not be finite.}. 

The notation used in definition \ref{dfn:value-func} is a bit abusive. The
symbol $\mathbb{E}_{\pi}$ means the expected value of the random variable $R$
(the rewards), given that the agent follows the policy $\pi$. Recall that the
random variable $R$ depends on the agent's actions.

Figure \ref{fig:violinplot}, way back in Chapter \ref{chapter:Motivation}, is
a graphical representation of $v_\pi (s)$ for the Miniopoly game for each square
in the board! Actually, it is not \textit{exactly} the value function for the
game Miniopoly, since the states of that game are determined not only by
\textit{where} on the board the player is but also on other factors such as
ownership of the square and how much money is available. The important thing is,
as we showed earlier, the value function can be estimated! This idea is key to
developing the algorithms that allow reinforcement learning to be used in
practice.

For the practical implementations of Reinforcement Learning, it is also handy to
consider how deviating from the chosen policy only at the present time and
follow it for all subsequent decisions affects the accumulated rewards. This can
help us evaluate if the current policy is on track to be the best available or
if we can find an action that would yield better results in the long run. To
evaluate the impacts on the value of a state by choosing action $a$ we introduce
the following function. 

\begin{dfn}{Action-Value Function}{action-value-func}
	We define the value of taking action $a$ in state $s$ at time $t$ and
	following policy $\pi$ for times $t+k$ once again as the discounted sum of
	accumulated rewards:
	\[
		q_\pi (s, a) \coloneqq \mathbb{E}_{\pi} \left[ \sum_{k=0}^{\infty} \gamma^{k} R_{t+k+1} \vertsep S_t = s, A_t = a \right].
	\]
\end{dfn}

One of the characteristic properties of the value functions we have introduced
this chapter, and indeed more generally, the problems that arise from optimal
strategies in \ac{mdp}s is that they can be written recursively. In other words, to
evaluate the value of a given state; we can ``play it out'' and find that value
as a combination of the values of all the possible states, the agent will visit in
the future.

\begin{lemma}{Recursive evaluation}{recursive-eval}
	The value function $v_\pi$ can be written in terms of itself as
	\[
		v_\pi (s) = \sum_{a \in \Actions} \pi(a \mid s) \sum_{s', r} p(s', r \mid s, a) \left[ r + \gamma \, v_\pi (s') \right].
	\]
\end{lemma}

Lemma \ref{lem:recursive-eval} will be the basis for the methods we will review
further in this thesis. What may not be obvious is that there may exist a
trivially-evaluable value function. The agent's last action can only come from a
relatively small list of possible actions; the computational burden of exploring
said list is often realistic. The second to last action taken comes from a much
more extensive list of possible actions. However, we do not have to explore
every combination since we already have the values computed for each terminal
action.  Using the lemma, we can efficiently compute the value for each
state-action pair for the second-to-last step in the process and so on for the
$n$th-to-last. In simple terms, we can evaluate by ``starting from the end'' and
playing back to the beginning. This relationship was first noted by Bellman
\cite{bellman1957}, one of the pioneers of \textit{dynamic programming}.

\begin{proof}[Proof of Lemma \ref{lem:recursive-eval}]
	We define:
	\[
		G_t \coloneqq \sum_{k=t+1}^{\infty} \gamma^{k-t-1} R_k = R_{t+1} + \gamma R_{t+2} + \gamma^{2} R_{t+3} + \cdots, 
	\]
	to make the definition of $v_\pi (s)$ more compact.
	\begin{align}
		v_\pi (s) &= \mathbb{E}_\pi \left[ G_t \vertsep S_t = s \right] \nonumber \\
		&= \mathbb{E}_\pi \left[ R_{t+1} + \gamma G_{t+1} \vertsep S_t = s \right] \nonumber \\
		\label{eq:paso-2}
		&= \mathbb{E}_\pi [R_{t+1} \mid S_t = s] + \gamma \mathbb{E}_\pi \left[ G_{t+1} \vertsep S_{t} = s \right].
	\end{align}

	To calculate the expected value of $R_{t+1}$ with the functions defined
	earlier, we use the following argument. We need to calculate
	\[
		\mathbb{E}_\pi [R_{t+1} \mid S_t = s] = \sum_{r} r \cdot \IP{R_{t+1} = r \mid S_t = s},
	\]
	but the probability density on the right hand side is not immediately
	available. We calculate it by using the definition of conditional
	probability and marginalizing probabilities:
	\begin{align}
		p(s', r \mid s, a) &\coloneqq \IP{S_{t+1} = s', R_{t+1} = r \mid S_t = s, A_t = a}, \nonumber \\
		\pi(a \mid s) p(s', r \mid s, a) &= \IP{S_{t+1} = s', R_{t+1} = r, \, A_t = a \mid S_{t} = s}, \nonumber \\
		\sum_{a} \pi(a \mid s) p(s', r \mid s, a) &= \IP{S_{t+1} = s', R_{t+1} = r \mid S_{t} = s}, \nonumber \\
		\sum_{a, s'} \pi(a \mid s) p(s', r \mid s, a) &= \IP{R_{t+1} = r \mid S_{t} = s} , \nonumber\\
		\sum_{a} \pi(a \mid s) \sum_{s'} p(s', r \mid s, a) &=  \IP{R_{t+1} = r \mid S_{t} = s} \nonumber \\
		\sum_{a} \pi(a \mid s) \sum_{s', r} p(s', r \mid s, a) \, r &= \mathbb{E}_\pi \left[ R_{t+1} \vertsep S_t = s \right]. \label{eq:cacho-uno}
	\end{align}

	We can find $\mathbb{E}_\pi [G_{t+1} \mid S_t = s]$ by similar arguments.
	The transition from state $s$ to state $s'$ given that an action $a$ was
	taken happens with probability
	\[
		\sum_a \pi(a \mid s) \sum_{s', r} p(s', r \mid s, a).
	\]
	Therefore, by Markov's property \eqref{eq:markov-sequence},
	\begin{align}
		\label{eq:cacho-dos}
		\mathbb{E}_\pi [G_{t+1} \mid S_t = s] &= \sum_a \pi(a \mid s) p(s', r \mid s, a) \mathbb{E}_\pi \left[ G_{t+1} \mid S_{t+1} = s' \right], \nonumber \\
		&= \sum_a \pi(a \mid s) p(s', r \mid s, a) v_\pi (s').
	\end{align}

	Combining \eqref{eq:cacho-uno} and \eqref{eq:cacho-dos}, making a substitution in \eqref{eq:paso-2} and factoring, we get:
	\begin{equation}
		\label{eq:bellman-equation-v}
		v_\pi (s) = \sum_a \pi(a \mid s) \sum_{s', r} p(s', r \mid s, a) \left[ r + \gamma v_\pi (s') \right].
	\end{equation}
Equation \eqref{eq:bellman-equation-v} is called Bellman's equation for the
value function.

The results in probability and stochastic processes used in this proof are
briefly reviewed in appendix \ref{appendix:Stochastic}.
\end{proof}

\section{Striving for the best}

As discussed, we are looking for a policy that achieves the maximum possible value
of the value function of a given state, denoted $v_*(s)$ and defined as
\[
	v_* (s) \coloneqq \max_{\pi} v_\pi (s),
\]
for all $s$ in $\States$. Notice how $v_* \in \R$. Similarly, we can also
define the optimal action-value function, denoted $q_*$, as
\[
	q_* (s, a) \coloneqq \max_{\pi} q_\pi (s, a),
\]
for all $s \in \States, a \in \Actions(s)$. The value $q_*$ represents the
expected return for taking action $a$ in state $s$ at time $t$ and then
following the optimal policy for $t+k$. Formally this can be written as
\[
	q_{*} (s, a) = \mathbb{E} \left[ R_{t+1} + \gamma v_* (S_{t+1}) \vertsep S_t = s, A_t = a \right],
\]
Which can be further simplified by using our previously defined state-value function as:
\begin{equation}
	\label{eq:q-by-v}
	q_\pi (s, a) = r(s, a) + \gamma \sum_{s'} p(s', r \mid s, a) \, v_\pi (s),
\end{equation}
for any policy $\pi$. In particular, for the optimal state-value and action-value functions, we have:
\begin{equation}
	q_*	(s, a) = r(s, a) + \gamma \sum_{s'} p(s', r \mid s, a) \, v_* (s).
\end{equation}

The Recursive Evaluation Lemma \ref{lem:recursive-eval} gives a recursive
relationship any value function must satisfy. However, $v_*$ is special. It can
be written in a special form \cite{bellman1957,SuttonBarto,raoRL4F} without
having to reference any specific policy as:
\begin{equation}
	\label{eq:prop-dependencia-vq}
	v_* (s) = \max_{a \in \Actions(s)} q_* (s, a) \quad \forall s \in \States.
\end{equation}
This relationship is called \textit{Bellman's optimality equation} for the value
function. Intuitively, this relationship is very natural. It states that the
value of a state under an optimal policy must equal the expected return for the
best action from that state \cite[Ch.~3.6]{SuttonBarto}.

Since the state-value and action-value functions are intimately related to one
another we can express the idea behind Bellman's Optimality equation in terms of
the action-value function as well
\[
	q_* (s, a) = \sum_{s', r} p(s', r \mid s, a) \left[ r + \gamma \max_{a'} q_{*} (s', a') \right].
\]

Bellman's optimality equation for the value function is, in fact, a system of
equations, one for each state $s$. The system of equations can be solved
explicitly when considering a deterministic policy, and the number of states is
small. However, it becomes computationally inefficient or simply intractable
when the number of states grows because of the curse of dimensionality, as the
\ac{rl} problem is combinatoric in nature. Solving the system of equations
explicitly is equivalent to listing out every possible combination of states and
actions.

\begin{remark}{The Curse of Dimensionality}
	By ``curse of dimensionality'', we refer to the fact that in combinatorial
	problems such as listing all the action-value pairs of the \ac{rl} problem,
	whenever the number of states or actions grows linearly, the combinations
	grow exponentially or factorially.
\end{remark}

Other methods to solve Bellman's equations are part of an area called dynamic
programming. Dynamic programming refers to a collection of algorithms and
heuristics that can be used to solve problems with a similar structure to the
\ac{rl} problem. Dynamic programming (DP) is prevalent in areas such as
Operations Research and other disciplines in which the nature of the problems is
episodic and demands having the ability to adapt \textit{dynamically}. DP
problems are often, as in this case, practically impossible to solve exactly in
a computationally efficient way.

Most of the methods implemented in practical applications to solve \aclp{mdp}
and in the implementations of Reinforcement Learning are based on techniques
that stem from the application of DP concepts. In contrast, this thesis focuses
on a technique that avoids DP completely to try instead to leverage the
properties of other optimization problems. Namely, the ability to compute
efficiently.

\part{Approximate solutions to the Reinforcement Learning Problem}
\label{part:II}

	\section{Introduction}

As discussed in previous chapters \ref{chapter:ReinforcementLearning}, the
problem of finding the best policies by using Bellman's optimality equations
falls within the realm of dynamic programming. The problem is that even if an
explicit solution can be given under certain conditions, the computational
burden of calculating exact solutions is often too significant to overcome, even
on modern computing equipment. And given the ``curse of dimensionality'', even
if the contemporary problems became tractable in the future thanks to the
ever-increasing computing power, that very same improvement in computing power
would motivate researchers to tackle bigger still problems. So, to use the RL
techniques, we must find a way to use our computational resources more
efficiently. 

Since Reinforcement Learning happens to be part of the techniques often grouped
under the umbrella of ``artificial intelligence'', it has enjoyed much attention
for decades. Thanks to these research efforts and numerous applications in the
industry, there are several battle-tested approximate solutions to the RL
problem we have developed here. Among these are: Q-learning, Monte--Carlo
estimation methods, temporal difference learning, and many others that are
described in detail in \cite[Chapter~4]{SuttonBarto}. The somewhat novel
technique described in this thesis was developed in the first decade of the
2000s and is not part of the standard toolbox for solving RL problems since it
was developed initially in the area of management science and particularly for
the problem of probabilistic inventory management, following the previous work,
laid out since the 1980s. 

Specifically, the technique to be laid out in this part of the thesis was
developed by \citeauthor*{farias2003LP2ADP}, as a continuation of previous work
laid out by \citeauthor*{denardo1970} and \citeauthor*{depenoux1963} in
\cite{denardo1970} and \cite{depenoux1963} respectively. In a nutshell,
\citeauthor*{farias2002thesis} casts the dynamic programming problem that arises
from solving Bellman's optimality equations as a linear program and then gets
around the curse of dimensionality by using linear approximations for the
interest functions to reduce the number of variables in the problem. Without
further ado, let us get to the details right after developing the necessary
background.

\section{Exact Dynamic Programming}
In Chapter \ref{chapter:ReinforcementLearning}, we showed that using Bellman's
optimality equations, we can obtain optimal policies if we have access to the
optimal value ($v_*$) or action-value ($q_{*}$) functions. We denote these
functions underscored by $*$ to accentuate the fact that they are optimal in the
sense that they satisfy Bellman's optimality equations, which are
\eqref{eq:bellmans-value} and \eqref{eq:bellmans-action-value} for $v_*$ and
$q_*$ respectively. 

% TODO: Make sure que estas ecuaciones ya estén justificadas antes y esto solo sea recordatorio 

\begin{equation}
\label{eq:bellmans-value}
v_{*}(s) = \max_{a} \sum_{s', r} p(s', r \mid s, a) \left[ r + \gamma v_{*} (s')
\right],
\end{equation}
\begin{equation}
\label{eq:bellmans-action-value}
q_{*}(s, a) = \sum_{s', r} p(s', r \mid s, a) \left[ r + \gamma \max_{a'} q_{*}
(s', a') \right],
\end{equation}
for all $s \in \States, a \in \Actions$.

Equation \eqref{eq:bellmans-value} is important, but not that helpful in
practice. It tells us the best possible reward a learning agent can expect in a
learning task. In other words, the best possible reward \textit{after} the agent
followed the best policy but tells us very little about how to obtain that
policy. We would like to predict the total reward a given policy will
yield. Thankfully, in chapter [CITE], we obtained an expression to calculate the
expected reward a given policy $\pi$ will yield starting from a certain state
$s$, that we will refer to as \textit{Bellman's recurrence equation} from now
on. 

\begin{equation}
\label{eq:bellmans-recurrence}
% TODO: Transcribir (4.4) de S&B.
v_\pi (s) = \sum_{a \in \Actions} \pi(a \mid s) \sum_{s', r} p(s', r \mid s, a) \left[ r + \gamma v_{\pi}(s') \right].
\end{equation}

Using \eqref{eq:bellmans-recurrence} we can obtain an exact solution for $v_\pi$
solving one equation for each state $s \in \States$. This is often called the
\textit{prediction problem} in the literature \cite[Chapter~4.1]{SuttonBarto}.
Transforming equation \eqref{eq:bellmans-recurrence} we obtain the following,
simpler expression:

\begin{align}
\label{eq:bellmans-recurrence-prime}
v_\pi(s) &= \sum_{a} \pi(a \mid s) \sum_{s', r} p(s', r \mid s, a) \left[ r + \gamma v_\pi (s') \right] \nonumber \\
&= \sum_{a} \pi(a \mid s) \left[ \sum_{s', r} r p(s', r \mid s, a) + \gamma \sum_{s', r} v_\pi (s') p(s', r \mid s,a) \right] \nonumber \\
&= \underbrace{\sum_{r} r \sum_{s'} p(s', r \mid s, a)}_{r(s,a) (dfn. \ref{dfn:rewards-func})} \sum_a \pi(a \mid s) + \gamma \sum_{s', r} v_\pi (s') p(s', r \mid s, a) \sum_{a} \pi(a \mid s) \nonumber \\
&= \sum_{a} \pi(a \mid s) \left[ r(s,a) + \gamma \sum_{s'} p(s' \mid s, a) v_\pi (s') \right] \tag{\ref{eq:bellmans-recurrence}'}.
\end{align}

Since \eqref{eq:bellmans-recurrence-prime} defines one equation for each state,
we can take the hint one step further and think of the system of equations
defined in terms of vectors and matrices. This approach will later serve to
define Bellman's policy\footnote{or ``one-step'' \cite[pg.~9]{nadeemward2021},
or ``expectation backup'' \cite[Lect.~3, Contraction Mapping]{silver2015}, the
literature uses various names. We follow
\cite{rao2022}.} and optimality operators, a powerful tool.

For now, let us consider the value function as a vector, following
\cite[pg.~132]{raoRL4F}. Specifically,
\begin{equation*}
    % FIXME: No entiendo al 100 porqué ya no tienen sub-pi.
    \vec{v} = \left[ v (s_1), \dots , v (s_{|\States|}) \right].
\end{equation*}
Recall that $r(s, a)$ is the expected reward upon taking action $a$ being in
state $s$, and $p(s' \mid s, a)$ is the probability of transitioning from state
$s$ to state $s'$ after taking action $a$. We define:
\begin{align*}
    R_\pi (s) &= \sum_{a \in \Actions} \pi(a \mid s) \, r(s,a), \\
    P_\pi (s, s') &= \sum_{a \in \Actions} \pi(a \mid s) \sum_{s' \in \States} p(s' \mid s, a),
\end{align*}
which result from a slightly rearranged version of
\eqref{eq:bellmans-recurrence-prime}.

We denote by $\vec{R}_\pi$ the vector $\left[ R_\pi(s_1), \dots, R_\pi
(s_{|\States|}) \right]$ and $\vec{P}_\pi$ the stochastic matrix $\left[
P_\pi(s_i, s_{i'}) \right]$ that defines the transition probabilities from any
state $s_i$ to every other distinct state $s_{i'}$. As promised, this is the key
to what will become one of our most powerful tools: Bellman's policy operator.
This will enable us to study how \textit{any} $v$ acts on the set of states
$\States$.

\begin{dfn}{Bellman's Policy Operator}{}
    We denote by $T_\pi$ the operator $T_\pi: \R^{|\States|} \to \R^{|\States|}$
    defined as:
    \begin{equation*}
        T_\pi \vec{v} = \vec{R}_\pi + \gamma \vec{P}_\pi \vec{v},
    \end{equation*}
    for any value function vector $\vec{v} \in \R^{|\States|}$. A
    better-motivated definition of this operator can be found in
    \cite[Ch.~5.4]{raoRL4F}.
\end{dfn}

Using the newly defined Bellman's Policy Operator, we can rewrite Bellman's
recurrence equation \eqref{eq:bellmans-recurrence} as
\begin{equation*}
    T_\pi \vec{v}_\pi = \vec{v}_\pi.
\end{equation*}
This means $\vec{v}_\pi$ is a fixed point of the Bellman Policy Operator.

Seizing the built-up momentum, let us write Bellman's optimality equation for
the value function \eqref{eq:bellmans-value} using another operator.

\begin{dfn}{Bellman's Optimality Operator}{}
    We denote by $T_*: \R^{|\States|} \to \R^{|\States|}$ \emph{Bellman's Optimality Operator}, defined as:
    % FIXME: Aqui change r(s,a) va multiplicado con vector de 1's.
    \begin{equation*}
        (T_{*} \vec{v})(s) = \max_{a \in \Actions} \left\{ r(s, a) + \gamma \sum_{s' \in \States} p(s' \mid s, a) \vec{v}(s') \right\}. 
    \end{equation*}
\end{dfn}

Once again, rewriting \eqref{eq:bellmans-value} using the Optimality Operator we
find that,
\begin{equation}
    \label{eq:bellmans-optimality-operators}
    T_* \vec{v}_{*} = \vec{v}_{*}.
\end{equation}
In other words, the optimality operator $T_*$ is a non-linear operator with a
fixed point $\vec{v}_*$ \cite[Lect. Jan 15 2019]{rao2022}.

As promised, Bellman's operators yield the solutions to both the optimal value
function and the prediction problem, which come in handy to prove whether or
not our RL problem has solutions or under which circumstances. However, so far,
we have no idea how to solve them.

With our toolbox almost complete, it is time to advance our search for
optimality. As previously mentioned, there are several approaches to solving
Bellman's equations are based on dynamic programming, yielding several
algorithms we don't review in detail in this thesis as they are outside of
scope. The literature for those techniques is excellent. For a more detailed
treatment of said algorithms, please review \cite{SuttonBarto} and
\cite{raoRL4F}.

\subsection{Approaching optimality}
So far, our goal has been to find the ``best'' policy, but what does it mean for
a policy to be the best? We can not compare policies directly, but we can
compare the value function's value for each of them. The optimal, the
``best'' policy is the one that maximizes the value function.

We say that a given policy $\pi$ is in a certain sense \textit{better} than the
policy $\pi'$, which we write as $\pi \geq \pi'$, whenever $v_\pi(s) \geq
v_{\pi'} (s)$ for every $s \in \States$. This is called a partial ordering over
the space of policies. Even better, the equality is only satisfied when both
policies are optimal, as stated in the following lemma.

\begin{lemma}{}{equality-on-optimality}
    For any two optimal policies $\pi^{*}_{1}$ and $\pi^{*}_{2}$, for all $s \in
    \States$ they evaluate to the same on the value function. That is,
    $v_{\pi^{*}_{1}} (s) = v_{\pi^{*}_{2}} (s)$.
\end{lemma}

{
    % FIXME: Resolver esto
    \bfseries
    \centering
    Pongo la demostración?? Aqui o en los apéndices??
}

Wielding lemma \ref{lem:equality-on-optimality} we can prove that there always
exists an optimal policy for the RL problem we have been studying.

\begin{thrm}{}{opt-policy-existence}
    For a Reinforcement Problem based on a discrete-time, finite-space Markov Decision Process, the following hold:
    \begin{itemize}
        \item There exists an optimal policy $\pi_*$. That is, $v_{\pi_*} (s)
            \geq v_{\pi}$ for all other policies $\pi$ and all states $s \in
            \States$.
        \item All optimal policies achieve the optimal value function given by
            \eqref{eq:bellmans-value}. That is, $v_{\pi_*}(s) = v_* (s)$ for all
            $s \in \States$ where $\pi_*$ is one optimal policy.
        \item All optimal policies achieve the optimal action-value function,
            given by \eqref{eq:bellmans-action-value}.
    \end{itemize}
\end{thrm}

\begin{proof}
    We follow the proof in \cite[Pg.~115]{raoRL4F} closely.

    As a consequence of lemma \ref{lem:equality-on-optimality}, we only need to
    find a policy that maximizes both the value and action value functions,
    achieving the maximum values: $v_*$ and $q_*$ respectively.

    A policy that is a candidate to be optimal can be constructed as follows:
    \begin{equation*}
        \pi_{*}^{D} (s) = \argmax_{a} q_{*} (s, a) \quad \forall s \in \States.
    \end{equation*}

    For now, we assume that the resulting action $a$ for any given state $s$ is
    unique.

    We show that $\pi_{*}^{D}$ maximizes the optimal value and action-value
    functions. As established in chapter \ref{chapter:ReinforcementLearning},
    equation \eqref{eq:prop-dependencia-vq}, $v_* (s) = \max_a q_* (s, a)$, and
    therefore,
    \begin{equation*}
        v_* (s) = q_* (s, \pi_{*}^{D}(s)).
    \end{equation*}

    In other words, we maximize the value function if we first take the action
    prescribed by the policy $\pi_{*}^{D}$ followed by the action from the next
    time steps that maximize the value function, and so on. This is precisely
    why Bellman's recurrence equations work as they do. Likewise, the
    action-value function is maximized following whatever action $\pi_{*}^{D}$
    prescribes. Formally,
    \begin{equation*}
        q_{\pi_{*}^{D}} (s, a) = q_* (s,a) \quad \forall s \in \States, a \in \Actions.
    \end{equation*}

    Lastly, we show that $\pi_{*}^{D}$is an optimal policy by contradiction.

    Suppose that $\pi_{*}^{D}$ is not an optimal policy. That means that there
    exists some other policy $\pi$ such that $\pi > \pi_{*}^{D}$. In the partial
    order defined previously, that means that $v_{\pi} (s) > v_{\pi_{*}^{D}}
    (s)$ for some state $s$!. This contradicts the definition of the
    optimal value function $v_* (s) = \max_\pi v_\pi (s)$ for all $s$. Therefore
    $\pi_{*}^{D}$ must be an optimal policy.
\end{proof}

A much more exciting and aesthetically pleasing way of proving theorem \ref{thrm:opt-policy-existence} is by using the properties of Bellman's operators.

Theorem \ref{thrm:opt-policy-existence} guarantees the existence of an optimal
policy in theory. But we are interested in finding such a policy in practice,
preferably in a way that is computable this century. 

\begin{proof}
    Aqui se usa el operador de Bellman y se muestra que es contracción, luego se
    aplica el teorema de punto fijo de Banach.
\end{proof}


\section{Approximate Linear Programming}
Having established the necessary context, we can at last address the main
objective of this Chapter, and indeed the whole thesis: avoiding the use of
inefficient dynamic programming and use linear programming to solve Bellman's
equations.

First, we note that by Theorem 6.2.2a in \cite[Ch.~6.9.1]{puterman2014},
whenever a given policy $\vec{v}$ satisfies
\begin{equation*}
    \vec{v} \geq T_\pi \vec{v} = \vec{R}_\pi + \gamma \vec{P}_{\pi} \vec{v}
\end{equation*}
for all possible policies, then, by Bellman's optimality equation
\eqref{eq:bellmans-optimality-operators},
\begin{equation*}
    \vec{v} \geq \vec{v}_* = T_* \vec{v}_*.
\end{equation*}
Where vectors $\vec{v}$ and $\vec{v}_*$ are compared component-wise. In other words, we can find an upper bound $\vec{v} \geq T_{*} \vec{v}$ by considering the following set of linear constraints:
\begin{equation}
    \vec{v} \geq r(s, a) + \gamma \sum_{s'} p(s' \mid s, a) \vec{v}(s') \quad \forall a \in \Actions, s \in \States.
\end{equation}

Since the solution to \eqref{eq:bellmans-optimality-operators} is guaranteed to
exist by Theorem \ref{thrm:opt-policy-existence} and be such that $\vec{v}_* =
T_* \vec{v}_*$, we can approach it by finding the smallest vector upper bound
given that we can generate upper bounds as discussed just before.

Finding the smallest upper bound leads to the following Linear Program:
\begin{equation}
\label{lp:exact-lp}
\tag{ELP}
\begin{array}{rl@{}ll}
    \displaystyle \min_{\vec{v} \in \R^{|\States|}} & \vec{c}^{\top} \vec{v} (s) \\
    \text{Subject to} & \vec{v} (s) \geq r(s,a) + \gamma \sum_{s'} p(s' \mid s, a) \, \vec{v} (s') & \quad \forall a \in \Actions, s \in \States. \\
    & \vec{v}(s) \text{ unconstrained,} & \quad \forall s \in \States.
\end{array}
\end{equation}

Professor \citeauthor{farias2002thesis} refers to Linear Program
\eqref{lp:exact-lp} as the \textit{exact Linear Program}
\cite[Ch.~2.3]{farias2002thesis}. This LP (linear program) has $|\States|$
variables and $|\States| \times |\Actions|$ constraints. This makes it
tremendously vulnerable to the curse of dimensionality. Solving this LP for the
number of states in a modern RL problem becomes prohibitively computationally
expensive, even when the average-case complexity of solving an LP is [NUMERO Y
REFERENCIA], much better than $\mathcal{O}(|\States|!)$ [IGUAL REFERENCIA].

To surmount this prohibitive cost, we turn to one of math's most proliferous
ideas: linear approximations.

\subsection{Linear everything}

Approximating complicated functions by collections of linear or linear-affine
functions is not new. One of the most proliferous applications of this idea is
linear regression in the area of statistical learning, or its more glamorous
name: machine learning. The main idea behind linear regression is approximating
a relationship between a vector of \textit{regressors}, $\vec{x}_i$, measured
attributes observed for some phenomenon of interest; and a vector of response
variables $\vec{y}$. Specifically, the relationship to be estimated is the
conditional expectation of $\vec{y}$ given the $\vec{x}_i$, or $\mathbb{E} \left[ \vec{y}
\mid \vec{x}_i \right]$. In a nutshell, the relationship is modeled by a linear
function $\vec{x}_{i}^{\top} \vec{B}$. The matrix $\vec{B}$ is filled with
coefficients that minimize the distance between the approximation and the actual
observed values $\vec{x}_i$. The key idea is that these parameters are obtained
by measuring the error between the predicted linear relationship and the true,
observed values. The exact nature of this process is not important at the
moment, but it is important to keep in mind that it can only be carried out if
we have access to the actual observed values of interest $\vec{x}_i$. For our
problem of interest, we have no such luxury. We have to be a bit more clever.

\subsubsection{Linear architecture}
The strategy followed in \cite{farias2002thesis} is generating scoring functions
with a pa\-ra\-me\-trized class of functions, similar to linear regression but
carrying out the ``fitting'' without being able to sample the function we are
trying to approximate.

This parametrized class of functions will be a basis for the space of value
functions made of linear functions $\varphi_i : \States \to \R$ for $i = 1,
\dots, K$, with $K \ll |\States|$ a set parameter. Following the standard
notation used in statistics, we denote $\widehat{\vec{v}} \approx \vec{v}$ the
linear approximation to the value function.

\begin{dfn}{Basis matrix}{}
    We define a matrix $\Phi \in \R^{|\States| \times K}$ given by:
    \begin{equation*}
        \Phi =
        \begin{bmatrix}
            | & \vdots & | \\
            \varphi_1 & \cdots & \varphi_K \\
            | & \vdots & |
        \end{bmatrix}.
    \end{equation*}
\end{dfn}

This way, use the representation for $\widehat{\vec{v}} = \Phi \vec{\beta}$,
where $\vec{\beta}$ is a vector of parameters that will fit this representation.
Armed with this one last tool, let us take another look at the exact LP
\eqref{lp:exact-lp}, substituting the function for the approximation wherever
pertinent.

\begin{equation}
\label{lp:approx-lp}
\tag{ALP}
\begin{array}{rl@{}ll}
    \displaystyle \min_{\vec{\beta} \in \R^{K}} & \vec{c}^{\top} \Phi \vec{\beta} \\
    \text{S.t.} & \vec{\beta}\vec{v} (s) \geq r(s,a) + \gamma \sum_{s'} p(s' \mid s, a) \, \Phi \vec{\beta} (s') & \quad \forall a \in \Actions, s \in \States. \\
    & \vec{\beta} \text{ unconstrained}.
\end{array}
\end{equation}

The LP in \eqref{lp:approx-lp} is referred to as the \textit{approximate linear
program}. Notice that the number of variables in this LP reduced from
$|\States|$ to $K$. The number of constraints was not reduced, but according to
\cite{farias2002thesis}, most of them become inactive, and solutions can be
approximated efficiently with modern methods. The rest of
\citeauthor{farias2002thesis}'s work shows how the special structure inherited
from dynamic programming can be used to efficiently sample constraints, making
the solution of this LP more efficient.

The subsequent chapters of this part are dedicated to briefly reviewing the extensions
and results of this method.


\section{Bibliographical Notes}

For the development of Bellman's Optimality Operator, we follow several sources:
\begin{itemize}
    \item Bellman's original paper \cite{bellman1957}.
    \item Several lectures and corresponding notes:
    \begin{itemize}
        \item David Silver's Course \cite[Lects.~2-3]{silver2015}.
        \item Basic Reinforcement Learning course at McGill University
            \cite[Lect.~2]{moisescu-parejaa}.
        \item Foundations of Reinforcement Learning with Applications in Finance
            course notes by Ashwin Rao \cite[Lect. on Jan 15 2019]{rao2022}.
    \end{itemize}
    \item \citeauthor{nadeemward2021}'s thesis \cite{nadeemward2021} which
        summarizes brilliantly the dynamic programming-based numerical
        approaches we skip in this section.
\end{itemize}

For a more thorough development of why Bellman's Operator is defined, please
consult any of the sources referenced directly above.

The conceptual leap involved in casting the dynamic programming problem as a
linear program using the properties of Bellman's operators, entire chapters are
dedicated to explaining the background and motivating the ideas, and laying a
solid foundation in \ref{puterman2014}, specifically chapters 2 and 6.

Sadly, we are constrained in scope and can not possibly lay the entire
groundwork for this idea to feel more natural and better motivated. For the
interested reader, \citeauthor{puterman2014}'s book is a great, if involved
read and is cited by several other bibliographic pieces used in this chapter.
	\label{chapter:PropertiesGuarantees}
As previously mentioned, the Approximate Linear Programming (ALP) method of
solving the RL problem is based on work by \citeauthor{farias2003LP2ADP}. This
next chapter is devoted to reproducing and discussing some of the main results
presented in \cite{farias2003LP2ADP}. The proofs are left out in favour of more
verbose explanations of what the results might entail. This chapter aims to
demistify some of the claims made about linear program \eqref{lp:approx-lp}
being ``easier'' to solve.

Recall that in the definition of \eqref{lp:approx-lp} we referred to the vector
$\vec{c}$ as the \emph{state-relevance wights}. The choife of state-relevance
weights does not influence the solution of \eqref{lp:exact-lp}, but it does
affect \eqref{lp:approx-lp}. The results discussed below demonstrate the impact
on the quality of the resulting approximation.

\section{Preliminaries}

\begin{dfn}{Vector-Weighted $\ell_1$ norm}{vectorw-l1-norm}
    The \emph{vector-weighted} 1-norm over the space $\ell_1$, denoted $\left\| \cdot \right\|_{1, \vec{c}}$, of a vector $\vec{x}$ is defined as
    \begin{equation*}
        \left\| \vec{x} \right\|_{1, \vec{c}} \coloneqq  \sum_i |x_i| c_i.
    \end{equation*}
\end{dfn}

To measure the quality of a specific policy $\pi$ we will consider the how the
value $v_\pi(s)$ compares to the optimal $v_* (s)$ when the initial state $s$ is
random variable with probability distribution $\sigma$. Intuitively, how far are
the expected total discounted rewards from the optimal when following policy
$\pi$.

\begin{dfn}{Expected increase in value following $\pi$}{expected-value-increase}
    The expected increase in value following a policy $\pi$ is defined as
    \begin{equation}
        \label{eq:expected-value-increase}
        \E_{s \sim \sigma} \left[ v_* (s) - v_\pi (s) \right] = \left\| \vec{v
        }_* - \vec{v}_\pi \right\|_{1, \sigma}.
    \end{equation}
    The notation $s \sim \sigma$ means that $s$ is a particular realization of the
    random variable $S$, which is distributed according to $\sigma$.
\end{dfn}

The attentive reader might have noticed a slightly abusive notation: $\| \cdot
\|_{1, \sigma}$. We defined the weighted $\ell_1$ norm for vectors, but $\sigma$
is not a vector. This is justified; since, as reviewed in chapter
\ref{chapter:ApproximateLinearP}, vectors $\vec{v}_*$ and $\vec{v}_\pi$ are
column vectors in which each entry is the state-value function evaluated at a
single state $s$. By extension, we can imagine that the vector $\sigma$ used in
\eqref{eq:expected-value-increase} is a column vector where each entry is the
probability of some $s \in \States$ happening. Formally, we define this
extension of the weighted norm as follows.

\begin{dfn}{Distribution-Weighted $\ell_1$ norm}{distributionw-l1-norm}
    Extending the vector-weighted $\ell_1$ norm we define the norm
    \begin{equation*}
        \left\| \vec{v} \right\|_{1, \sigma} \coloneqq \sum_{s \in \States} \sigma(s) \, |v(s)|,
    \end{equation*}
    for some vector $\vec{v} \sim \sigma$ where $\sigma$ is a probability
    distribution.
\end{dfn}

Next, we define a probability measure that captures the probability of the agent
being in some state given that it is following policy $\pi$ and starte on some
randomly distributed $s \sim \sigma$.

\begin{dfn}{The $\mu$ measure}{mu-measure}
    We define a measure $\mu$ as
    \begin{equation*}
        \mu_{\pi, \, \sigma}^{\top} \coloneqq (1 - \gamma) \sigma^{\top} \sum_{t=0}^{\infty} \gamma^{t} \vec{P}_{\pi}^{t}.
    \end{equation*}
    Since $\sum_{t=0}^{\infty} \gamma^{t} \vec{P}_{\pi}^{t} = (I - \gamma
    \vec{P}_\pi)^{-1}$, where $I$ is the identity matrix, we have
    \begin{equation*}
        \mu_{\pi, \, \sigma}^{\top} = (1 - \gamma) \sigma^{\top} (I - \gamma \vec{P}_{\pi})^{-1}.
    \end{equation*}
    We say $\mu$ is a measure in the sense of measure theory. It can be shown
    \Cite[pg.~864]{farias2003LP2ADP} that $\mu$ is a probability distribution.
\end{dfn}

\section{Results}

We begin with a lemma that helps illustrate the role of state-relevance weights
for the approximation procedure.

\begin{lemma}{}{farias-vanroy-lem1}
    A vector $\vec{\beta}_0$ solves the following LP
    \begin{equation*}
    \begin{array}{rl@{}ll}
        \displaystyle \min_{\vec{\beta} \in \R^{K}} & \vec{c}^{\top} \Phi \vec{\beta} \\
        \text{S.t.} & \displaystyle \Phi \vec{\beta} (s, a) \geq r(s, a) + \gamma \sum_{s'} p(s' \mid s, a) \Phi \vec{\beta}(s') & \quad \forall a, s \in \Actions, \States . \\
    \end{array}
    \end{equation*}
    if and only if it solves
    \begin{equation*}
    \begin{array}{rl@{}ll}
        \displaystyle \max_{\vec{\beta} \in \R^{K}} & \left\| \vec{v}_* - \Phi \vec{\beta} \right\|_{1, \vec{c}} \\
        \text{S.t.} & \displaystyle \Phi \vec{\beta} (s, a) \geq r(s, a) + \gamma \sum_{s'} p(s' \mid s, a) \Phi \vec{\beta}(s') & \quad \forall a, s \in \Actions, \States . \\
    \end{array}
    \end{equation*}
\end{lemma}

Lemma \ref{lem:farias-vanroy-lem1} corresponds to Lemma 1 in
\Cite[Pg.~853]{farias2003LP2ADP}, and effectively establishes that
\eqref{lp:approx-lp} can be solved as the maximization of a weighted norm where
the weights are given by the state-relevance weights. The state-relevance
weights vector does what it's name suggests: impose a balance of the quality of
the approximation of the value function for each state. This means that by
tuning $\vec{c}$ we can give more or less attention to different regions of the
state space $\States$.

Since the state-relevance vector imposes a restriction on how good our
approximations can be. How good an approximation can we hope to find? The
following theorem establishes bounds in the quality of approximations.

\begin{thrm}{}{farias-vanroy-thm1}
    Let $\vec{v}$ such that $\vec{v} \geq T^{*} \vec{v}$, then
    \begin{equation*}
       \left\| \vec{v}_{*} - \vec{v}_{\pi_v} \right\| \leq \frac{1}{1 - \gamma} \left\| \vec{v}_{*} - \vec{v} \right\|_{1, \mu_{\pi_v, \, \sigma}}.
    \end{equation*}
\end{thrm}

We use the notation $\pi_v$ to mean the policy that yields the value $v$. This
policy can be extracted once a specific $v$ is known. Theorem
\ref{thrm:farias-vanroy-thm1} assures us that if the approximate value function
$\vec{v}$ found by solving the LP is close to the optimal $\vec{v}_{*}$, then
the performance of the policy generated by $\vec{v}, \pi_v$  will also be close
to the perfomance achieved by the optimal policy. By combining theorem
\ref{thrm:farias-vanroy-thm1} with lemma \ref{lem:farias-vanroy-lem1} we
conclude that we would like the initial state-relevance weight vector to capture
the (discounted) frequency with which different states are expected to be
visited. In other words we would like for $\vec{c}$ to be as close as possible
to $\mu_{\pi_v, \, \sigma}$. That is, we would like to invest more effort
approximating the function for the states the learning agent is most likely to
visit, compromising on worse approximations for infrequent states.

\subsection{Error bounds for the ALP}

\part{Applications to Supervised Machine Learning}
\label{part:III}

	\appendix

	\chapter{Simulating Miniopoly}
\label{appendix:MiniopolySim}

\section*{Code used to simulate the game described on earlier 
chapters.}

The specific rules for the game described in Chapter 
\ref{chapter:motivation} can be best visualized as literal 
code. Listing \ref{lst:main-logic} contains the actual code 
used to check and enforce buying and renting rules, as well as 
carry out transactions. All code presented here is valid Julia 
code.

\lstinputlisting[language=julia, firstline=161, lastline=200, 
caption=Main logic for the game, 
label={lst:main-logic}]{../codigo/simulacion-miniopoly/Miniopoly.jl}

The code above depends on the definition of several ``Objects'' 
in Object Oriented Programming parlance: 
\lstinline{GameManager}, \lstinline{Player} and 
\lstinline{Square}.

The \lstinline{Player} Structure represents a single player of 
the game. The code in listing \ref{lst:player-def} shows what 
attributes or fields characterize it (i.e. the information we 
keep track of). Notice for instance the attribute 
\lstinline{rewardslog}, which is a mapping between a 
\lstinline{Square} represented as its number and the its 
ownership status to a number. This is the mechanism that keeps 
track of the reward obtained by buying or not a certain square.

\lstinputlisting[language=julia, firstline=6, lastline=10, 
caption=Definition of Player structure, 
label={lst:player-def}]{../codigo/simulacion-miniopoly/MiniopolyPieces.jl}

The \lstinline{Square} Structure represents a purchasable 
square as defined by the game rules outlined on Chapter 
\ref{chapter:motivation}. 

\lstinputlisting[language=julia, firstline=52, lastline=57, caption=Definition of Square structure, label={lst:square-def}]{../codigo/simulacion-miniopoly/MiniopolyPieces.jl}

The \lstinline{GameManager} Structure is essentially what keeps 
track of the property ledger, the current state of the game, 
among other things. Listing \ref{lst:gamemanager-def} shows 
what information is kept for book-keeping purposes.

\lstinputlisting[language=julia, firstline=57, lastline=62, 
caption=Definition of GameManager structure, 
label={lst:gamemanager-def}]{../codigo/simulacion-miniopoly/Miniopoly.jl}


\nocite{*}
\cleardoublepage
\printbibliography

\end{document}