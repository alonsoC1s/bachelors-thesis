\documentclass[colorful]{sty/itam-thesis}

\usepackage{lipsum}

\usepackage[
		showframe,
		paperwidth=17cm,
		paperheight=22.5cm,
		nofoot=true,
		bindingoffset=1.1cm,
		inner=1.6cm,
		outer=1.8cm,
		top=2.5cm,
		bottom=1.5cm
	]{geometry}

% Bibliografía
\usepackage{csquotes}
\usepackage[
	backend=biber,
    citestyle=alphabetic,
    style=alphabetic,
    maxcitenames=2,
    ]{biblatex}
\addbibresource{refs.bib}

% Paquete custom
\usepackage{sty/thesis-package}

\author{Alonso Martinez Cisneros}
\title{Titulo}
\date{2022}

\begin{document}

\frontmatter
\maketitle
\makefrontmatter
\makeacknowledgements{
	Thanks to
}{
	Acknowledgements to
}

\mainmatter

%====================================================%
% Start of the content proper ---------------------- %
%====================================================%
\part{Primera parte}
\chapter{Hola}

\lipsum[1-10]

\section{probando}
Hola hola

\chapter{DOS}

Adios

\begin{equation}
	f: \R \to \R, \iint
\end{equation}

Tete

\section{hai}

\begin{mytheo}{Label}{Perron-F}
	Holis
\end{mytheo}

\begin{lstlisting}[language=julia, caption=Aplicando algoritmo de cifrado]
# Definiendo sistema de Lorenz
function lorenz!(du, u, p, t, σ=10, β=8//3, ρ=28)
    du[1] = σ * (u[2] - u[1])
    du[2] = u[1] * (ρ - u[3]) - u[2]
    du[3] = u[1] * u[2] - β * u[3]
end
\end{lstlisting}

\nocite{*}
\printbibliography

\end{document}
